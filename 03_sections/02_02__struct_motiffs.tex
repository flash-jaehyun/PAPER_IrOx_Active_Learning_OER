% %%%%%%%%%%%%%%%%%%%%%%%%%%%%%%%%%%%%%%%%%%%%%%%%%%%%%%%%%
% | - Structural analysis of IrO2 and IrO3 oxides %%%%%%%%%
% %%%%%%%%%%%%%%%%%%%%%%%%%%%%%%%%%%%%%%%%%%%%%%%%%%%%%%%%%
% TODO: Synthax for calling subplots in figure Figure 1.d or Figure 1.d)
% TODO: "we note that ..." is used a few times Kirsten
%
% Important Points:
%   * A good candidate set is a set of materials that have a large degree of 'structural diversity'
%
%   * We can make an analogy between the shape of these plots and typical van-der Waals curves
%     * I would argue that the tails are much less steep than a van-der Waals curve because in most cases the there is still a large degree of bonding
%       * Essentially these systems can create more and more 'porous' structures as the density is decreases, which allows us to get really large porous structures
%         * A good point can be made here that these types of systems could be useful for battery applications
%           * Pores are good in this context I think?
%
%
%   * The main coordination environments are 6 and 4-fold coordinated (Ir-O4/6 units)
%     * This is consistent with crystal field theory, or literature, etc.
%     * 6/4-coord accounts for TEMP percent of the IrO2/3 candidate space
%       * TODO Get these numbers
%     * There are in addition to 6/4-coord, a lot of other types of coordination environments
%       * A lot of these are simply weird, aphysical structures for sure
%         * Unassociated oxygens
%           * Singly-associated (part of N-hedra)
%           * Completely un-unassociated O atoms, just randomly placed in unit cell
%
%         * Especially at the extreme ends of the average coord metric
%       * But there are also a lot of legitimate coordination environments
%         * I've manually parsed the dataset and found
%
%   * There is a lot to say about the different kinds of 100% corner sharing octahedral IrO3 systems
%     * There are a lot of these systems that differ in subtle ways
%       * Basically octahedra can be "rotated" in different directions, making them distinct
%
%   * Interesting types of systems
%     * Layered systems (A lot of variety here)
%     * Cubic coordination environments (A few cool looking examples)
%     * New coordination environments
%       * TODO Remind myself again of these
%
%   * IrO2 has more 4-fold coordinated systems than IrO3
%     * Makes sense, the more oxygens you have the more oxygen-rich motifs are favored
%   * IrO2 has large "dip" in the EvsV "convex hull" while IrO3 has a much more shallow increase in energy as you move to the right from the most stable polymorph
%     * This is probably due to the fact that IrO3 can create more porous layered structures
%
% NOTES:
%   * Describe convex hull, classes of structures (\ce{$\alpha$-AlF3} like, rutile like, and layered, should be segregated in hull plot)
%   * Describe structures within each class, cite lit where appropriate
% __| %%%%%%%%%%%%%%%%%%%%%%%%%%%%%%%%%%%%%%%%%%%%%%%%%%%%%

% %%%%%%%%%%%%%%%%%%%%%%%%%%%%%%%%%%%%%%%%%%%%%%%%%%%%%%%%%
% ####################### Paragraph #######################
% %%%%%%%%%%%%%%%%%%%%%%%%%%%%%%%%%%%%%%%%%%%%%%%%%%%%%%%%%
% TEMP
% %%%%%%%%%%%%%%%%%%%%%%%%%%%%%%%%%%%%%%%%%%%%%%%%%%%%%%%%%
% | - Paragraph start
% There should be a paragraph which discussed the structural drift and performance/acceleration in more detail
% I would use the updated version of Figure 2c.
% #########################################################

As a next step we assess the stability as well as the structural variety of the DFT optimized structures, consisting of 381 \IrOtwo and 185 \IrOthree unique polymorphs, showing the enthalpy of formation plotted versus the volume per atom in Figure~\ref{fig:E_vs_V} a and b.
%
To obtain a physically meaningful cutoff for the enthalpy of formation above which materials are unlikely to be synthesizable, we computed the metastability limits for \IrOtwo and \IrOthree relative to their amorphous phases using the methodology of Persson \latin{et.~al.}~\cite{Aykol2018} (see SI for more details on methodology).
%
The metastability limits were computed to be -0.33 and -0.34 eV/atom for \IrOtwo and \IrOthree, respectively,
and are displayed as horizontal lines in Figure~\ref{fig:E_vs_V}a and b. There are 195 and 74 polymorphs for \IrOtwo and \IrOthree, respectively, that are within the meta-stability line, and are therefore potential candidates for synthesizable polymorphs.
%
% These last two sentences can be removed
%These values are almost identical, possibly indicating that the energy of the amorphous phases are insensitive to the exact stoichiometry.
%
%Measuring the metastability relative to the hull for each stoichiometry yields a metastability window of 0.5 and 0.3 eV/atom for \IrOtwo and \IrOthree, respectively.
% __|

% ####################### Paragraph #######################
% %%%%%%%%%%%%%%%%%%%%%%%%%%%%%%%%%%%%%%%%%%%%%%%%%%%%%%%%%
% There is a large variety in density on this plot
% IrO3 has a small dependance on the volume
% %%%%%%%%%%%%%%%%%%%%%%%%%%%%%%%%%%%%%%%%%%%%%%%%%%%%%%%%%
% | - Paragraph start
A large variety in structure density and coordination environmen is desirable, since this will increase the likelihood of discovering novel materials. We note that for both compositions, there is a large variety in density such that both low volume (dense) and high volume (porous) structures are contained in the data set. The lowest volume (most dense) structures corresponds to an atomic packing factor of roughly 0.50, where the most stable structures are found. On the other end, the highest volume (most porous) systems sampled have atomic packing factors going as low as 0.15.
%where the shape of envelope curve (lowest energy for a given volume) is analogous to the shape of the Lennard-Jones potential. \textbf{is this important or expected?}
%
% NOTE Can make a comment about the fact that these porous species may be over-stabilized with PBE DFT
However, for \IrOthree there is a comparatively weak relationship between the energy and volume,
such that even highly porous structure are within 0.1 eV of the most stable phase. We note that the stability of highly porous systems has a tendency to be overestimated on the DFT+GGA level, where van der Walls interactions should be included for a more accurate description. \cite{}. % What is the error in estimated energy?  

%
% COMBAK Explain this point better, this has to do with the oxidation state, coordination preservation rules that I've been playing around with

%This property is likely due to the fact that \IrOthree's oxidation state can more can form readily form layered or porous structures.
%
% \textbf{unclear what you mean here, vdw is attractive so shouldn't it just make them more stable?}

%Although formation energies for highly porous systems tend to be over stabilized at the DFT+GGA level of theory because of the improper description of van der Walls interactions, the magnitude of these errors \textbf{don't severely impact the results or something}
% __|

% ####################### Paragraph #######################
% %%%%%%%%%%%%%%%%%%%%%%%%%%%%%%%%%%%%%%%%%%%%%%%%%%%%%%%%%
% Explaining coordination motiff distribution (octahedral, tetrahedral
% %%%%%%%%%%%%%%%%%%%%%%%%%%%%%%%%%%%%%%%%%%%%%%%%%%%%%%%%%
% | - Paragraph start
%
The structures have been classified with respect to the Ir-O coordination environment (such as octahedral, square pyramidal, tetrahedral, cubic, etc.),
by using the chemEnv package, developed by Waroquiers et. al. \cite{Waroquiers2017} and implemented in pymatgen \cite{Ong2013}.
%
Although the dataset was found to contain a large range of coordination environments ranging from 2 to 10,
structures with a coordination number of 6 (octahedreal) or 4 (tetrahedreal) were found to be most prevalent
%(accounting for roughly TEMP percent of all structures)
and have been highlighted in blue and red respectively in Figure~\ref{fig:E_vs_V}a and b.
%
For both \IrOtwo and \IrOthree the vast majority of most stable (within 0.1 eV) compounds adopt an octahedral coordination environment, a common coordination motif found to be favorable in many other transition metal oxides.\cite{Waroquiers2017}
%
% This has to be expanded upon or else maybe dropped, we can discuss
% I don't understand the end of this sentence.  Give some information of how this applies to your structures.
% I've added a sentence at the end where we simply describe whether our most stable structures are corner vs edge sharing
The arrangement of the octahedreal units, which are connected through either corner-, edge sharing,
can furthermore be used to classify the structures, that typically has a combination of the two.
%
% Look into this deeper, there is probably a difference in fraction of mixed corner+edge between IrO2 and IrO3
Of the top 10 \IrOtwo and \IrOthree structures TEMP of them have a mixed corner and edge sharing packing environments while the rest are corner sharing.
% __|

% ####################### Paragraph #######################
% %%%%%%%%%%%%%%%%%%%%%%%%%%%%%%%%%%%%%%%%%%%%%%%%%%%%%%%%%
% TEMP
% %%%%%%%%%%%%%%%%%%%%%%%%%%%%%%%%%%%%%%%%%%%%%%%%%%%%%%%%%
% | - Paragraph start
In Figure~\ref{fig:E_vs_V}c and d, a selection of meta-stable structures is shown for \IrOtwo and \IrOthree respectively.
For \IrOtwo, we found the rutile structure to be the most stable, in correspondence with the Inorganic Crystal Structure Database (ICSD) where only the rutile and pyrite phases are reported to have been synthesized \cite{bolzan1997structural, shirako2014synthesis}, the pyrite phase also predicted to be metastable in this work. Additionally, several common \ABtwo crystal structures were found within the dataset, including columbite\cite{columbite}, brookite\cite{brookite} and anatase\cite{anatase} phases (not all shown).
%
For \IrOthree the five most stable systems were shown in Figure~\ref{fig:iro2_al},and have been labeled with numbers (1)-(5) in Figure~\ref{fig:E_vs_V} b. However, we have identified several additional meta-stable structures, including 2D(i), highly porous(iii) and 1D(v) polymorphs with varying degrees of porosity and connectivity,
which could be interesting for battery applications \cite{}.
%
% \textbf{the labeling scheme jumping between two figures is a little confusing here, and it obfuscates the point I think you are trying to make in this last sentence}
% COMBAK
Interestingly, the most stable \IrOthree structure in MP~\cite{mp-1097041} corresponds to the Nth most stable \IrOthree polymorph in our dataset (Cmcm space group and structure (iv) in Figure~\ref{fig:E_vs_V}d), meaning that we have uncovered N \IrOthree structures more stable than anything currently known.
% __|


% | - Figure | Energy vs. Volume (motif distribution)
\begin{figure*}[!htb]
\centering
\makebox[\textwidth][c]{\includegraphics[
width=\textwidth,height=\textheight,keepaspectratio]
{02_figures/e_vs_v_motiffs.pdf}
}
\caption{\label{fig:E_vs_V}
%381 \IrOtwo and 185
Enthalpy of formation for the \num{381} \IrOtwo (a) and \num{185} \IrOthree (b) DFT optimized structures in the candidate data set plotted against the volume per atom.
%
Insets in the low energy region for (a) and (b) are shown.
%
The color bar represents the average coordination number between Ir and O, with 6 (octahedra) and 4 (tetrahedra) highlighted.
%
(c) and (d), select \IrOtwo and \IrOthree polymorph structures.
}
\end{figure*}
% __|
