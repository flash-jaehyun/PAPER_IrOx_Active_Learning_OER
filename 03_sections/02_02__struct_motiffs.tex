%%%%%%%%%%%%%%%%%%%%%%%%%%%%%%%%%%%%%%%%%%%%%%%%%%%%%%%%%%%%%%%%%%%%%%%%%%%%%%%
% Structural analysis of IrO2 and IrO3 oxides %%%%%%%%%%%%%%%%%%%%%%%%%%%%%%%%%
% %%%%%%%%%%%%%%%%%%%%%%%%%%%%%%%%%%%%%%%%%%%%%%%%%%%%%%%%%%%%%%%%%%%%%%%%%%%%%
% TODO: Synthax for calling subplots in figure Figure 1.d or Figure 1.d)
%
% Important Points:
%   * A good candidate set is a set of materials that have a large degree of 'structural diversity'
%
%   * We can make an analogy between the shape of these plots and typical van-der Waals curves
%     * I would argue that the tails are much less steep than a van-der Waals curve because in most cases the there is still a large degree of bonding
%       * Essentially these systems can create more and more 'porous' structures as the density is decreases, which allows us to get really large porous structures
%         * A good point can be made here that these types of systems could be useful for battery applications
%           * Pores are good in this context I think?
%
%
%   * The main coordination environments are 6 and 4-fold coordinated (Ir-O4/6 units)
%     * This is consistent with crystal field theory, or literature, etc.
%     * 6/4-coord accounts for TEMP percent of the IrO2/3 candidate space
%       * TODO Get these numbers
%     * There are in addition to 6/4-coord, a lot of other types of coordination environments
%       * A lot of these are simply weird, aphysical structures for sure
%         * Unassociated oxygens
%           * Singly-associated (part of N-hedra)
%           * Completely un-unassociated O atoms, just randomly placed in unit cell
%
%         * Especially at the extreme ends of the average coord metric
%       * But there are also a lot of legitimate coordination environments
%         * I've manually parsed the dataset and found
%
%   * There is a lot to say about the different kinds of 100% corner sharing octahedral IrO3 systems
%     * There are a lot of these systems that differ in subtle ways
%       * Basically octahedra can be "rotated" in different directions, making them distinct
%
%   * Interesting types of systems
%     * Layered systems (A lot of variety here)
%     * Cubic coordination environments (A few cool looking examples)
%     * New coordination environments
%       * TODO Remind myself again of these
%
%   * IrO2 has more 4-fold coordinated systems than IrO3
%     * Makes sense, the more oxygens you have the more oxygen-rich motifs are favored
%   * IrO2 has large "dip" in the EvsV "convex hull" while IrO3 has a much more shallow increase in energy as you move to the right from the most stable polymorph
%     * This is probably due to the fact that IrO3 can create more porous layered structures
%
% NOTES:
%   * Describe convex hull, classes of structures (\ce{$\alpha$-AlF3} like, rutile like, and layered, should be segregated in hull plot)
%   * Describe structures within each class, cite lit where appropriate
%%%%%%%%%%%%%%%%%%%%%%%%%%%%%%%%%%%%%%%%%%%%%%%%%%%%%%%%%%%%%%%%%%%%%%%%%%%%%%%


% ################################# Paragraph #################################
% %%%%%%%%%%%%%%%%%%%%%%%%%%%%%%%%%%%%%%%%%%%%%%%%%%%%%%%%%%%%%%%%%%%%%%%%%%%%%
% TEMP
% %%%%%%%%%%%%%%%%%%%%%%%%%%%%%%%%%%%%%%%%%%%%%%%%%%%%%%%%%%%%%%%%%%%%%%%%%%%%%
% | - Paragraph start
% There should be a paragraph which discussed the structural drift and performance/acceleration in more detail
% I would use the updated version of Figure 2c.
%
Here we assess the structural variety of the dataset consisting of 448 \IrOtwo and 258 \IrOthree unique polymorphs,
where the relation between the enthalpy of formation and the inverse atomic density (Volume/atom) is shown in Figure~\ref{fig:E_vs_V} a) and b).
%
The metastablilty limit of ?? and ?? for \IrOtwo and \IrOthree respectively is shown in the plot,
where several meta-stable polymorphs have been identified.
% __|


% ################################# Paragraph #################################
% %%%%%%%%%%%%%%%%%%%%%%%%%%%%%%%%%%%%%%%%%%%%%%%%%%%%%%%%%%%%%%%%%%%%%%%%%%%%%
% TEMP
% %%%%%%%%%%%%%%%%%%%%%%%%%%%%%%%%%%%%%%%%%%%%%%%%%%%%%%%%%%%%%%%%%%%%%%%%%%%%%
% | - Paragraph start
We note that for both compositions,
there is a large variety in density,
such that both low volume (dense) and high volume (porous) structures are represented.
%
As expected, the most stable systems are found at low volume,
where the shape of envelope curve (lowest energy for a given volume) is analogous to the shape of the Lennard-Jones potential.
%
However, for \IrOthree in particular,
there is a rather weak relation between the energy and the volume,
where highly porous structure are found to be meta-stable.
% __|


% ################################# Paragraph #################################
% %%%%%%%%%%%%%%%%%%%%%%%%%%%%%%%%%%%%%%%%%%%%%%%%%%%%%%%%%%%%%%%%%%%%%%%%%%%%%
% TEMP
% %%%%%%%%%%%%%%%%%%%%%%%%%%%%%%%%%%%%%%%%%%%%%%%%%%%%%%%%%%%%%%%%%%%%%%%%%%%%%
% | - Paragraph start
%
The structures have been classified with respect to the Ir-O coordination environment
(such as octahedral, square pyramidal, tetrahedral, cubic, etc.),
by using the chemEnv package, developed by Waroquiers et. al. \cite{Waroquiers2017} and implemented in pymatgen \cite{Ong2013}.
%
Although the dataset was found to contain a large range of coordination environments, structures with a coordination number of 6 (octahedreal) or 4 (tetrahedreal) were found to be most prevalent and have been highlighted in blue and red respectively in Figure~\ref{fig:E_vs_V} a) and b).
%
For both \IrOtwo and \IrOthree the most stable candidates are seen to have an octahedreal oxygen coordination, which is generally favored by many other metal oxides. \cite{Waroquiers2017}
%
The arrangement of the octahedreal units, which are connected through either corner-, edgesharing, can furthermore be used to classify the structures, that typically has a combination of the two. 
% __|


% ################################# Paragraph #################################
% %%%%%%%%%%%%%%%%%%%%%%%%%%%%%%%%%%%%%%%%%%%%%%%%%%%%%%%%%%%%%%%%%%%%%%%%%%%%%
% TEMP
% %%%%%%%%%%%%%%%%%%%%%%%%%%%%%%%%%%%%%%%%%%%%%%%%%%%%%%%%%%%%%%%%%%%%%%%%%%%%%
% | - Paragraph start
%
In Figure~\ref{fig:E_vs_V} c) and d) a selection of meta-stable structures is shown for \IrOtwo and \IrOthree respectively.
%
For \IrOtwo, we found rutile to be the most stable,
in correspondence with experiment \cite{}.
% TODO Make active voice
Also, several of the well-know crystal structures, including  columbite, purite, brookite and anatase (not all shown) are found to be meta-stable.
% 
For \IrOthree the five most stable systems were shown in Figure~\ref{},
and have been labeled with numbers (1)-(5) in Figure~\ref{fig:E_vs_V} b).
%
However, we have identified several additional meta-stable structures, including 2D(i), highly porous(iii) and 1D(v) polymorphs,
which could be interesting for battery applications \cite{}.
%
We note that the Cmcm polymorph in d) iv. is currently the lowest energy structure on Materials Project
(cite https://materialsproject.org/materials/mp-1097041/) \cite{}.
% __|






% | - Figure | Energy vs. Volume (motif distribution)
\begin{figure*}[!htb]
\centering
\makebox[\textwidth][c]{\includegraphics[
width=\textwidth,height=\textheight,keepaspectratio]
% {02_figures/e_vs_v_motiffs.png}
{02_figures/e_vs_v_motiffs.pdf}
}
\caption{\label{fig:E_vs_V}
%
Enthalpy of formation for the \num{448} \IrOtwo (a) and \num{258} \IrOthree (b) structures in the candidate data set plotted against the volume per atom.
%
%Color overlays indicate the dominant coordination motifs as indicated by the legend.
%
%Select polymorphs systems are displayed around the plot area.\textbf{this caption needs to be fleshed out}
}
\end{figure*}
% __|




% Additional esoteric coordination environments were identified manually, see SI.
%
%The resulting distribution is included in Figure~\ref{fig:E_vs_V}, which plots the formation enthalpy and volume, both normalized on a per atom basis. \textbf{this section needs to be fleshed out still and connections made between discovered structures and the literature, this seems like the place in the document that needs the most work}\\


% | - __old__

% | - Figure | IrO2 Convergence Plot
% \begin{figure*}
% \centering
% \makebox[\textwidth][c]{\includegraphics
%   {02_figures/ml_convergence_plots/00_master__iro2-ml-conv_v1__200dpi__0__outplot.png}
%   % {02_figures/ml_convergence_plots/iro2_ml_conv.png}
%   }
% \caption{\label{fig:convergence_plot_iro2_0}
% Gaussian process machine learning models trained initially on (a) publicly available DFT data for \IrOtwo and (b) all of the acquired DFT calculations from the active learning algorithm.
% See SI for additional panels at intermediate iterations of the active learning algorithm.
% The Gibbs formation energy (either DFT derived or predicted from the GP model) and associated GP estimated error (2 sigmas or something TEMP) is plotted for each polymorph in the \IrOtwo candidate space.
% The data points in each subset are ordered from most to least stable (lowest to largest DE formation).
% The individual markers are colored based on their ordering in the final converged GP model.
% Acquired structures are identified by their red borders and slightly larger size.
% The insets show the most stable TEMP structures, where several well known crystal structures are labeled.
% }
% \end{figure*}
% __|

% | - Figure | IrO3 Convergence Plot
% \begin{figure*}
% \centering
% \makebox[\textwidth][c]{\includegraphics
% {02_figures/ml_convergence_plots/00_master__iro3-ml-conv_v6__200dpi__0__outplot.png}
% % {02_figures/ml_convergence_plots/iro3_ml_conv.png}
% }
% \caption{\label{fig:convergence_plot_iro3_0}
% TEMP.
% }
% \end{figure*}
% __|

% __|
