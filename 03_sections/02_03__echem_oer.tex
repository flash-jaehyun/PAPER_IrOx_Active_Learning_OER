% %%%%%%%%%%%%%%%%%%%%%%%%%%%%%%%%%%%%%%%%%%%%%%%%%%%%%%%%%
% | - Electrochemical OER Application
% %%%%%%%%%%%%%%%%%%%%%%%%%%%%%%%%%%%%%%%%%%%%%%%%%%%%%%%%%
%
% __|
% %%%%%%%%%%%%%%%%%%%%%%%%%%%%%%%%%%%%%%%%%%%%%%%%%%%%%%%%%



% %%%%%%%%%%%%%%%%%%%%%%%%%%%%%%%%%%%%%%%%%%%%%%%%%%%%%%%%%
% | - Short Intro EChem Section
%
% __|
% %%%%%%%%%%%%%%%%%%%%%%%%%%%%%%%%%%%%%%%%%%%%%%%%%%%%%%%%%
% | - PARAGRAPH BODY
%
% TODO In the previous sections make sure to highlight the alpha and rutile IrO3 phases explicitly so that this sentence makes sense here.
We next performed \latin{ab-initio} thermodynamic analyses to test the OER electrochemical operational stability and activity of the most stable \IrOx in aqueous solution.
%
In particular, we compare the stability and activity of \rIrOtwo to our newly discovered \aIrOthree and \rIrOthree polymorphs.
%
In addition, we have also computed the stability and activity of a delithiated form of a recently reported $\beta$-Li\textsubscript{x}IrO\textsubscript{3} structure, referred here to as \bIrOthree.~\cite{Pearce2017,Pearce2019}
%
The OER activity was computed assuming the most common single-site associative OER mechanism utilizing the thermodynamic limiting potential analysis with the computational hydrogen electrode as described extensively in numerous previous works~\cite{Man2011,Rossmeisl2007,Kitchin2004,Bajdich2013}
(see also Supporting Information for details).
% __|
% %%%%%%%%%%%%%%%%%%%%%%%%%%%%%%%%%%%%%%%%%%%%%%%%%%%%%%%%%


% %%%%%%%%%%%%%%%%%%%%%%%%%%%%%%%%%%%%%%%%%%%%%%%%%%%%%%%%%
% | - Bulk Pourbaix Diagram
%
% __|
% %%%%%%%%%%%%%%%%%%%%%%%%%%%%%%%%%%%%%%%%%%%%%%%%%%%%%%%%%
% | - PARAGRAPH BODY
%
The calculated bulk Pourbaix diagram of the Ir-H\textsubscript{2}O system is shown in Figure~\ref{fig:bulk_pourbaix}.
%
The diagram was constructed by considering the thermodynamic equilibrium between the following species: Ir, \rIrOtwo, \aIrOthree, \rIrOthree, \bIrOthree, and an aqueous dissolved \IrOfourm species.
%
To obtain free energies, we utilized a free energy correction to our calculated values to reproduce the known experimental \DHf and \DGf of \rIrOtwo.~\cite{Barin1995}
%
While Ir and \rIrOtwo are most stable at low bias, \aIrOthree becomes the thermodynamically dominant phase under the relevant conditions for the OER (potentials $>$ \num{1.23} \VRHE and in an acidic environment).
%
The stability regions for the less stable \bIrOthree and \rIrOthree polymorphs (in the absence of other \IrOthree phases) are also included (unfilled solid lines).
%
It can be seen that these phases have a reduced, but sufficiently large, stability window relative to \IrOtwo and the \IrOfourm.
%
Removal of the \IrOthree phases from the bulk Pourbaix diagram results in a completely different thermodynamic picture of \IrOtwo stability (Figure~\ref{fig:bulk_pourbaix_wo_alpha}).
In total, we have discovered 21 unique \IrOthree polymorphs with a non-zero bulk Pourbaix stability region
(0 $\leq$ pH $\leq$ 14).
%
Interestingly, these 21 structures are more stable than the most stable \IrOthree structure in Materials Project~\cite{mp-1097041} (which is also present in our dataset, Figure~\ref{fig:E_vs_V}d.iv).
%
We note that thermodynamic driving force toward these \IrOthree phases under OER conditions suggests that these structures may spontaneously form under reaction conditions,
especially for systems that undergo a large degree of surface reconstruction~\cite{Seitz2016}.
% __|
% %%%%%%%%%%%%%%%%%%%%%%%%%%%%%%%%%%%%%%%%%%%%%%%%%%%%%%%%%


% =========================================================
% FIGURE ==================================================
% =========================================================
% | - Figure | Bulk Pourbaix Diagram
\begin{figure*}[!htb]
\centering
\makebox[\textwidth][c]{\includegraphics
% {02_figures/05_bulk_pourbaix/00_master__bulk-pourbaix__v6.pdf}}
{02_figures/05_bulk_pourbaix/00_master__bulk-pourbaix__v6__1200dpi.png}}
\caption{\label{fig:bulk_pourbaix}
%
Revised bulk Pourbaix diagram of the \ce{Ir}-\ce{H2O} system as a function of applied potential ($U_{SHE}$) and pH.
%
The diagram was constructed with Ir(s) (blue), \rIrOtwo (green), various \IrOthree polymorphs and a dissolved \IrOfourm ion species (dark grey).
%
The stability regions corresponding to the metastable \rIrOthree and \bIrOthree polymorphs (see text), in the absence of any competing \IrOthree phase, are displayed as yellow and pink lines, respectively.
%
The thermodynamic onset of OER (water equilibrium at \num{1.23} \VRHE) is also shown.
%
To be compared to Figure~\ref{fig:bulk_pourbaix_wo_alpha} without \IrOthree phases.
%
See Table \ref{table:bulk_pourb} for the bulk formation energies (\DGf) used to construct the diagram.
}
\end{figure*}
% __| =====================================================
% =========================================================


% %%%%%%%%%%%%%%%%%%%%%%%%%%%%%%%%%%%%%%%%%%%%%%%%%%%%%%%%%
% | - Introduction to OER Results
%
% __|
% %%%%%%%%%%%%%%%%%%%%%%%%%%%%%%%%%%%%%%%%%%%%%%%%%%%%%%%%%
% | - PARAGRAPH BODY
%
We next computed the surface energy Pourbaix plots and OER activity for various surface facets at select coverages (for simplicity we only choose bare and one monolayer of OH* and O*) of all four systems from Figure \ref{fig:bulk_pourbaix};
results are summarized in Figure~\ref{fig:oer_volcano}.
%
For each polymorph, surfaces were constructed by cleaving along the Miller indices with the highest calculated diffraction peaks, corresponding to planes with higher density of atoms.
%
The surface free energy Pourbaix plots identify which surface facets and surface coverage species are thermodynamically preferred under OER conditions.
%
Our results show that most of the facets prefer to have a high surface coverage of O*, therefore we consider mainly oxygen terminated surfaces for the OER analysis.
%
Our results are comparable to previous studies on the electrochemical stability of \IrOtwo surfaces~\cite{Nattino2019,Raman2020},
but without considering highly reconstructed facets such as (101).
%
The surface stability analysis is therefore crucial for accurate determination of the activity.
% __|
% %%%%%%%%%%%%%%%%%%%%%%%%%%%%%%%%%%%%%%%%%%%%%%%%%%%%%%%%%


% =========================================================
% FIGURE ==================================================
% =========================================================
% | - Figure | OER Volcano/Surface Pourbaix
% {02_figures/06_oer_results/00_oer_plot_v10__downsampled_0900x0900.pdf}}
% {02_figures/06_oer_results/00_oer_plot_v10__downsampled_0900x0900__v1.pdf}}
% {02_figures/06_oer_results/00_oer_plot_v10.pdf}}

\begin{figure*}
\centering
\makebox[\textwidth][c]{\includegraphics
% {02_figures/06_oer_results/00_oer_plot_v10__downsampled_0900x0900__v1.pdf}}
{02_figures/06_oer_results/00_oer_plot_v10__downsampled_0900x0900__600dpi.png}}
\caption{\label{fig:oer_volcano}
%
Summary of OER results for the following four bulk structures of \IrOx: \rIrOtwo (green), \aIrOthree (purple), \rIrOthree (orange), and \bIrOthree (pink).
%
(a) Surface energy Pourbaix diagrams for each structure, with the surface energy of various facets and coverages shown as a function of applied potential (\VRHE).
%
% The coverage with bare surfaces (light lines), *OH covered (medium lines) and *O covered surfaces (thick lines) are shown.
%
The bulk Pourbaix diagram's bounds of stability at pH \num{0} are superimposed as horizontal bars at the bottom of each subplot.
%
The pseudo-stability regimes for the metastable \bIrOthree and \rIrOthree are indicated by dashed vertical lines.
%
(b) OER activity volcano for \IrOx systems considered utilizing the \DGOmOH{} descriptor.
%
The horizontal lines correspond to recent experimental OER limiting potentials for \rIrOtwo (110)~\cite{Kuo2017} and SrIrO\textsubscript{3}~\cite{Seitz2016},
at \SI[mode=text]{10}{\mA\per\cm\squared} (extrapolated values).
%
(c) Corresponding structural models for selected OER surfaces at one mono-layer O* coverage used for calculation of the overpotentials.
%
Color legend: oxygen (red), purple (iridium), coordination motif (white).
%
Computational cell is displayed by black lines.
%
All OER slab models and corresponding DFT energies are freely available under the ``FloresActive2020''~\cite{upload_CatHub} dataset at Catalysis-hub.org~\cite{Winther2019}.
}
\end{figure*}
% __| =====================================================
% =========================================================


% %%%%%%%%%%%%%%%%%%%%%%%%%%%%%%%%%%%%%%%%%%%%%%%%%%%%%%%%%
% | - OER Volcano
%
% __|
% %%%%%%%%%%%%%%%%%%%%%%%%%%%%%%%%%%%%%%%%%%%%%%%%%%%%%%%%%
% | - PARAGRAPH BODY
%
The calculated OER activities of relevant OER stable surfaces are plotted against the \DGOmOH{} descriptor and are shown in Figure \ref{fig:oer_volcano}b.
%
There, we display two thermodynamic limiting potential volcanos based on (1) the standard universal~\cite{Man2011} (black) and (2) fitted (grey) scaling relations between the OER intermediates (Figure \ref{fig:scaling_relations}).
%
Additionally, we have also added a kinetic OER volcano (dashed line) from Dickens \latin{et al.}~\cite{Dickens2019} based on the detailed microkinetic model developed for rutile systems.
%
The kinetic volcano is constructed at the potential required to reach \SI[mode=text]{10}{\mA\per\cm\squared}.
%
The thermodynamic and kinetic volcanos agree remarkably well in the strong binding portion (left hand side) of the plot and exhibit similar optimum value, \DGOmOH{} $\approx$ \num{1.55}-\num{1.65} eV.
%
The corresponding surface structures for selected systems featuring high oxygen coverage are visualized in Figure \ref{fig:oer_volcano}c.
% __|
% %%%%%%%%%%%%%%%%%%%%%%%%%%%%%%%%%%%%%%%%%%%%%%%%%%%%%%%%%


% %%%%%%%%%%%%%%%%%%%%%%%%%%%%%%%%%%%%%%%%%%%%%%%%%%%%%%%%%
% | - TEMP

% __|
% %%%%%%%%%%%%%%%%%%%%%%%%%%%%%%%%%%%%%%%%%%%%%%%%%%%%%%%%%
% | - PARAGRAPH BODY
%
In general, the \rIrOtwo surfaces bind the OER intermediates relatively strongly, with theoretical limiting potentials of \mytilde\num{1.8} \VRHE (overpotential of \num{0.57} \VRHE) having the *O to *OOH potential limiting step, in agreement with previous theoretical studies.~\cite{Briquet2017,Strickler2019,Raman2020}
%
The predicted overpotentials of our \rIrOtwo systems are also within the range of experimentally observed values (horizontal lines).~\cite{Seitz2016, Kuo2017}.
%
The surfaces of the three \IrOthree polymorphs have \DGOmOH{} values shifted to higher energies, indicative of overall weaker binding energetics (see also Figure~\ref{fig:scaling_relations}).
%
On average, the adsorption of OH* and O* is weakened by \num{0.7} and \SI{1.2}{\electronvolt} relative to \IrOtwo (Table~\ref{table:oer_data}), respectively.
%
The highest performing systems include the \aIrOthree (100), (110), and (211), followed by \bIrOthree (101), and then \rIrOthree (110).
%
These surfaces have overpotentials of \mytilde\num{0.4} \VRHE,
which represents a \mytilde\num{0.2} \VRHE improvement over \rIrOtwo, mirroring the observed shift in experimental onset potentials (horizontal lines).~\cite{Seitz2016, Kuo2017}.
%
The primary driver for the improved OER activity is the higher oxidation state of \IrOthree compared to \IrOtwo, having only three $5d$-electrons for \ce{Ir^{6+}} as opposed to five $5d$-electrons in \ce{Ir^{4+}}, respectively.
%
Oxygen saturated \IrOthree systems thus bind OER intermediates more weakly, which leads to positive shift in \DGOmOH{}.
%
\IrOtwo and \RhOtwo are generally overbinding for OER~\cite{Dickens2019} so there is consequent improvement in OER activity when compared to these oxides.
%
These results are consistent with Back \latin{et al.}, who recently computed elevated activity in highly oxidized \IrOthree catalysts.\cite{Back2019}
%
An added feature of \aIrOthree is comparably higher density of active sites due to completely corner-sharing geometry.
%
The exact improvement in the theoretical overpotential is slightly dependent on the DFT level of theory and the inclusion of spin polarization, and has been discussed recently.~\cite{Briquet2017,Strickler2019}
% __|
% %%%%%%%%%%%%%%%%%%%%%%%%%%%%%%%%%%%%%%%%%%%%%%%%%%%%%%%%%
