%%%%%%%%%%%%%%%%%%%%%%%%%%%%%%%%%%%%%%%%%%%%%%%%%%%%%%%%%%%
% | - Electrochemical OER Application %%%%%%%%%%%%%%%%%%%%%
% %%%%%%%%%%%%%%%%%%%%%%%%%%%%%%%%%%%%%%%%%%%%%%%%%%%%%%%%%
%
% OER DFT modelling references:
%   * Man2011
%
% \chris{Im losing the big picture here at the end of the section, I think you are convoluting two different things that need to be broken up into their own paragraphs.  This OER activity part needs some work, let's discuss}
% __| %%%%%%%%%%%%%%%%%%%%%%%%%%%%%%%%%%%%%%%%%%%%%%%%%%%%%



% %%%%%%%%%%%%%%%%%%%%%%%%%%%%%%%%%%%%%%%%%%%%%%%%%%%%%%%%%
% ####################### Paragraph #######################
% %%%%%%%%%%%%%%%%%%%%%%%%%%%%%%%%%%%%%%%%%%%%%%%%%%%%%%%%%
% Short introduction into the catalysis section
% | - %%%%%%%%%%%%%%%%%%%%%%%%%%%%%%%%%%%%%%%%%%%%%%%%%%%%%
%
% TODO In the previous sections make sure to highlight the alpha and rutile IrO3 phases explicitely so that this sentence makes sense here.
We next performed \latin{ab-initio} thermodynamic simulations to elucidate the electrochemical operational stability of \IrOx and the OER activity of various \IrOthree phases previously computed.
%
In particular we chose to simulate the OER activity of the most stable \IrOtwo and \IrOthree polymorphs (rutile and our newly discovered \aIrOthree phase), the rutile-like \IrOthree polymorph.
%
In addition, we computed the OER activity of a delithiated form of a recently reported $\beta$-Li\textsubscript{x}IrO\textsubscript{3} (referred to here as \bIrOthree) structure, as it is a notable recent example of a experimentally synthsized \IrOthree system.~\cite{Pearce2017}
% __| %%%%%%%%%%%%%%%%%%%%%%%%%%%%%%%%%%%%%%%%%%%%%%%%%%%%%

% %%%%%%%%%%%%%%%%%%%%%%%%%%%%%%%%%%%%%%%%%%%%%%%%%%%%%%%%%
% ####################### Paragraph #######################
% %%%%%%%%%%%%%%%%%%%%%%%%%%%%%%%%%%%%%%%%%%%%%%%%%%%%%%%%%
% Revised Bulk Pourbaix Diagram
%
% Explain the Ir and IrO[4-] species
% QUESTION Use E or U for potential variable
% | - %%%%%%%%%%%%%%%%%%%%%%%%%%%%%%%%%%%%%%%%%%%%%%%%%%%%%
%
Figure~\ref{fig:bulk_pourbaix} reports the bulk \IrOx Pourbaix diagram (E vs. pH) constructed with the following species: Ir, \rIrOtwo, \aIrOthree,  \rIrOthree, \bIrOthree, and an aqueous dissolved \ce{IrO[4-]} species.
%
While Ir and \rIrOtwo are most stable at low bias, \aIrOthree becomes the thermodynamically dominant phase under the relevant conditions for the OER (potentials around \mytilde1.23 V vs. RHE and an acidic environment)
%
% Raul | I would like to make the more general point that IrO3 polymorphs are the stable species relative to IrO2 at OER conditions.
% I think that point needs to be made clearly before we dive into the detail of which particular polymorph is stable
% TEXT | This indicates that the \aIrOthree phase may be stabilized under the highly oxidizing conditions of the OER.
%
% I want to be careful about how I talk about the metastable structures in the bulk Pourbaix, they strictly speaking shouldn't be there at all
The stability regions of the metastable \rIrOthree and \bIrOthree phases (In the absence of any other \IrOthree polymorph) are indicated by unfilled solid lines, and show that these phases also have a large stability window relative to \IrOtwo and the Ir ion.
%
% TODO Create energy table for main bulk systems in SI
The similar formation energies (see Table \ref{table:oer_table}) for all three \IrOthree species suggest some or all of these \IrOthree phases may be present and are stable under OER conditions.


% | - __old__
% , in the absence of any other competing \ce{IrO_3} polymorphs, are indicated by unfilled solid lines.
%
% As shown, these metastable phases appear to also have a wide region of stability in the OER region,
%due to to their formation energies being within TEMP eV of the globally stable \aIrOthree phase, see SI table TEMP for the bulk energies of all considered phases.
%
%Because of their similar energies it is possible that some or all of these \ce{IrO_3} phases may be present and relevant for the OER.
%
%In the next section, we explore this possibility by computing the theoretical OER activity of these polymorph systems.
% __|

% __| %%%%%%%%%%%%%%%%%%%%%%%%%%%%%%%%%%%%%%%%%%%%%%%%%%%%%


% =========================================================
% FIGURE ==================================================
% | - Figure | Bulk Pourbaix Diagram
\begin{figure*}[!htb]
\centering
\makebox[\textwidth][c]{\includegraphics{02_figures/oer_activity_stability/bulk_pourbaix.pdf}}
\caption{\label{fig:bulk_pourbaix}
The revised Pourbaix diagram (electrochemical bulk phase stability) of the Ir-H$_2$O system as a function of applied potential (vs. SHE) and pH.
%
The most stable system studied (see Table XX in SI for a full list) are Ir-metal Ir(s) (blue), a \rIrOtwo  (green), and a dissolved \ce{IrO4[4-]} (gray).
%
These are compared to the \ce{IrO_3} polymorphs, \aIrOthree (purple), \rIrOthree (orange), and \bIrOthree (pink).
%
% I would like to add Srx IrO3 to show much less stable it is than alpha
The water equilibrium line at U\num{=1.23} V vs. RHE shows the ideal onset of OER.
%
% This plot highlights the importance of the \ce{IrO3} for OER operational stability.
}
\end{figure*}
% __|
% =========================================================


% %%%%%%%%%%%%%%%%%%%%%%%%%%%%%%%%%%%%%%%%%%%%%%%%%%%%%%%%%
% ####################### Paragraph #######################
% %%%%%%%%%%%%%%%%%%%%%%%%%%%%%%%%%%%%%%%%%%%%%%%%%%%%%%%%%
% Introduction to OER results
% | - %%%%%%%%%%%%%%%%%%%%%%%%%%%%%%%%%%%%%%%%%%%%%%%%%%%%%
Figure~\ref{fig:oer_volcano} summarizes the results of the electrochemical activity and surface stability analysis.
%
Figure~\ref{fig:oer_volcano}a reports the surface energy Pourbaix plots as a function of applied potential (at pH\num{=0}) for the four \IrOx crystals of interest.
%
The bulk phase limits of stability from Figure~\ref{fig:bulk_pourbaix} are included at the bottom of each subplot.
%
% TODO Insert XRD simulation into SI and reference here
For each polymorph, the specific facets were chosen from the highest intensity x-ray diffraction peaks from simulated powder-diffraction spectra\cite{Momma2011} (see Figure \ref{fig:xrd_patterns}),
as well as manually cleaving the bulks along intuitive facet planes.
%
For each facet we computed the surface free energy for three coverages, bare, *OH, and *O.
%
At modest overpotentials ($\eta$ \mytilde\num{0.3} or equivalently potentials of \mytilde\num{1.5} V vs. RHE) the convex hull is populated solely by oxygen terminated surfaces.
%
% COMBAK, Revise "mainly" if we include some different coverages
Consequently, we consider mainly oxygen terminated surfaces for the OER analysis.
% __| %%%%%%%%%%%%%%%%%%%%%%%%%%%%%%%%%%%%%%%%%%%%%%%%%%%%%

% %%%%%%%%%%%%%%%%%%%%%%%%%%%%%%%%%%%%%%%%%%%%%%%%%%%%%%%%%
% ####################### Paragraph #######################
% %%%%%%%%%%%%%%%%%%%%%%%%%%%%%%%%%%%%%%%%%%%%%%%%%%%%%%%%%
% OER results
% | - %%%%%%%%%%%%%%%%%%%%%%%%%%%%%%%%%%%%%%%%%%%%%%%%%%%%%
% Replace G_O and G_OH with my custom macros for consistency
The OER activity (expressed in terms of the limiting potential) for select oxygen terminated surfaces are shown in Figure~\ref{fig:oer_volcano} as a function of the $\Delta G_\mathrm{O} - \Delta G_\mathrm{OH}$ thermodynamic descriptor.
%
The two \rIrOtwo surfaces (100 and 110) bind the OER intermediates too strongly,
locating them at a theoretical limiting potential of \mytilde\num{1.8} V vs. RHE.
%
The predicted overpotentials of our \rIrOtwo systems are within the range of experimentally observed overpotentials found in literature.
%
% Reference all experimental IrO2 overpotentials I can find
The three \IrOthree polymorph surfaces all have a $\Delta G_\mathrm{O} - \Delta G_\mathrm{OH}$ descriptor towards the top and right of the volcano, indicative of weaker binding energetics.
%
% TODO How much weaker does IrO3 bind than IrO2 on average
This is evident from Figure~\ref{fig:scaling_relations}, which shows a clear distinction between the \IrOtwo and \IrOthree polymorphs, with \IrOthree binding on average 1.1 eV weaker than \IrOtwo.
%
The best performing systems, including the (100), (110), and (211) facets of \aIrOthree, \bIrOthree (101), and \rIrOthree (110), have overpotentials of \mytilde\num{0.4} V vs. RHE,
a \mytilde\num{0.2} V vs. RHE improvement over the \rIrOtwo system.
%
We note that the computed  overpotentials for our \rIrOtwo system differs from that reported in literature~\cite{Seitz2016} by \mytilde\num{0.2} V.
%
This discrepancy is due to our inclusion of spin-polarization in our \latin{ab-initio} calculations, which was neglected in Seitz \latin{et al.}.
%
For further details, we'll refer readers to the SI of a previous publication.\cite{Strickler2019}
% __| %%%%%%%%%%%%%%%%%%%%%%%%%%%%%%%%%%%%%%%%%%%%%%%%%%%%%


% =========================================================
% FIGURE ==================================================
% | - Figure | OER Volcano/Surface Pourbaix
\begin{figure*}
\centering
% \makebox[\textwidth][c]{\includegraphics{02_figures/oer_activity_stability/oer_volc_surf_pourb.png}}
\makebox[\textwidth][c]{\includegraphics{02_figures/oer_activity_stability/oer_volc_surf_pourb.pdf}}
\caption{\label{fig:oer_volcano}
% TODO Insert green band into figure, insert experimental references for this
%
Summary of OER results for the four bulk structures of \IrOx considered: rutile-\ce{IrO_2} (green), $\alpha$-\ce{IrO_3} (purple), rutile-\ce{IrO_3} (orange), and $\beta$-\ce{IrO_3} (pink).
%
(a) Surface energy Pourbaix diagrams for each structure, with the surface energy of various facets and coverages shown as a function of applied potential.
%
% COMBAK This may not be necessary, too much clutter on plots
The bulk Pourbaix diagram's bounds of stability at pH \num{0} are superimposed at the bottom of each subplot.
%
(b) OER activity volcano for \IrOx systems considered utilizing the \DGOmOH thermodynamic descriptor.
%
The purple dotted line corresponds to the experimental limiting potential at \num{10} mA cm\textsuperscript{-2} for \ce{IrO_3} \cite{Seitz2016},
while the green band corresponds to the range of experimentally observed overpotentials for pristine \ce{IrO_2} catalysts as reported in literature.
%
(c) Select surface facets for the four \IrOx crystal systems considered.
%
Color legend: oxygen (red), purple (iridium), coordination motif (white).
}
\end{figure*}
% __|
% =========================================================




% | - __old__
% Circles designate oxygen covered surfaces while triangles designate hydroxyl (*OH) terminated surfaces (relevant surface terminations were found via surface Pourbaix analyses).
% Surface energies at standard conditions (pH and V = 0) are reflected in the border color for each data point, where black indicates a low energy surface termination and white indicates more unstable surfaces.  % Currently not implemented as this
% The color range goes from x to y.
% __|
