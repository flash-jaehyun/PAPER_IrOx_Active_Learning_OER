%%%%%%%%%%%%%%%%%%%%%%%%%%%%%%%%%%%%%%%%%%%%%%%%%%%%%%%%%%%
% Abstract %%%%%%%%%%%%%%%%%%%%%%%%%%%%%%%%%%%%%%%%%%%%%%%%
% %%%%%%%%%%%%%%%%%%%%%%%%%%%%%%%%%%%%%%%%%%%%%%%%%%%%%%%%%
%
%%%%%%%%%%%%%%%%%%%%%%%%%%%%%%%%%%%%%%%%%%%%%%%%%%%%%%%%%%%


%regular Ch of M article does not have this restriction
%\comment{This is 248 words, needs to be 200 or less}
%
\noindent The discovery of high-performing and stable materials for sustainable energy applications is a pressing goal in catalysis and materials science.
%
Understanding the relationship between a material's structure and functionality is an important step in the process, such that viable polymorphs for a given chemical composition need to be identified.
%
Machine-learning based surrogate models have the potential to accelerate the search for polymorphs that target specific applications.
%
Herein, we report a readily generalizable active-learning (AL) accelerated algorithm for identification of electrochemically stable Iridium-oxide polymorphs of \IrOtwo and \IrOthree.
%
The search is coupled to a subsequent analysis of the electrochemical stability of the discovered structures for the acidic oxygen evolution reaction (OER).
%
The algorithm starts from all known AB$_2$ and AB$_3$ type structures (more than \num{38000}) by identifying 956 unique materials structural prototypes.
%
Next, using an active learning approach we are able to find \num{196} \IrOtwo polymorphs within the thermodynamic synthesizability limit and reaffirm the global stability of the rutile structure.
%
We find \num{75} synthesizable \IrOthree polymorphs and report a previously unknown \ce{FeF_{3}}-type structure as the most stable, termed as \aIrOthree.
%
To test model efficiency we compared to a random search via Density Functional Theory (DFT).
%
We report that using an AL approach we can increase at least two fold the rate of the discovery and acquires the most stable polymorphs in about \num{4.3} generations on average.
%
Analysis of the structural properties of discovered polymorphs reveals that octahedral local coordination environments are preferred for nearly all low energy structures.
%
Subsequent Pourbaix Ir-H$_2$O analysis shows that \aIrOthree is the globally stable solid phase under acidic OER conditions and supersedes the stability of rutile \IrOtwo.
%
Calculation of theoretical OER surface activities reveal ideal weaker binding of the OER intermediates on \aIrOthree than on any other considered Iridium-oxide.
%
We emphasize that the proposed AL algorithm  can be easily generalized to search for any binary metal-oxide structure with a predetermined stoichiometry.
