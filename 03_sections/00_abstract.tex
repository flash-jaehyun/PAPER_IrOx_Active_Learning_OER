%%%%%%%%%%%%%%%%%%%%%%%%%%%%%%%%%%%%%%%%%%%%%%%%%%%%%%%%%%%
% Abstract %%%%%%%%%%%%%%%%%%%%%%%%%%%%%%%%%%%%%%%%%%%%%%%%
% %%%%%%%%%%%%%%%%%%%%%%%%%%%%%%%%%%%%%%%%%%%%%%%%%%%%%%%%%
%
%%%%%%%%%%%%%%%%%%%%%%%%%%%%%%%%%%%%%%%%%%%%%%%%%%%%%%%%%%%



\comment{This is 248 words, needs to be 200 or less}
%
Materials science is primarily concerned with the underlying relationship between a material's structure and functionality,
where the knowledge of viable polymorphic forms of crystals plays an indispensable role.
%
Machine-learning based surrogate models have the potential to accelerate this process of creating the knowledge-base for materials polymorphs for target applications in under-explored chemistries.
%
Herein, we report on a readily generalizable active-learning (AL) accelerated algorithm for the targeted identification of novel and stable \IrOx (x=2 or 3) polymorphs and subsequent thermochemical analyses of the activity of these discovered structures towards the oxygen evolution reaction (OER).
%
We demonstrate that compared to a random search,
the AL framework more than doubles the efficiency of using DFT to find stable polymorphs out of a large array of prototypical structures.
%
We find nearly \num{195} \IrOtwo polymorphs within the thermodynamic synthesizability limit and reaffirm the rutile ground state.
%
For \IrOthree, we find \num{74} unique synthesizable polymorphs and report a previously unknown \ce{FeF_{3}}-like ground state.
%
The algorithm is exceptionally adept at quickly picking out the most stable polymorphs, with the most stable \aIrOthree phase discovered on average in only \num{4.3} generations.
%
An analysis of the structural properties of these metastable polymorphs reveals that octahedral local coordination environments are preferred for all low energy structures.
%
Subsequent Pourbaix and thermochemical analyses with this \aIrOthree phase show that it is fully stable under acidic OER conditions,
and delivers lower theoretical overpotentials compared to \rIrOtwo due to weaker and more ideal binding of the OER intermediates on its oxidized surfaces.





%| - MY OLD VERSION
% %
% Machine learning (ML) based surrogate models have become an increasingly common tool in the field computational materials discovery to overcome the relative expense of \latin{ab-initio} density functional theory (DFT).
% %
% Training such models on expansive DFT data sets has been used to discover new materials in the vast space of material compositions.
% %
% % COMBAK This is probably too strong
% However, surrogate model applications to the structural space of bulk crystals, \latin{i.e.} polymorphs, remain relatively unexplored.
% %
% Herein, we report on an active learning (AL) framework that searches for the most stable polymorph of a chemical space by utilizing a surrogate model which is sequentially updated on the fly with generated DFT data.
% %
% The model searches within a candidate data set of crystal motifs sourced from publicly available materials databases.
% %
% We demonstrate the efficacy of this AL-accelerated methodology by discovering the most stable polymorphs of \IrOtwo and \IrOthree within the candidate space with DFT calculations of only a fraction of the candidate space.
% % 7 out of the 10 most stable crystal structures for \IrOtwo and \IrOthree with less than 50 DFT calculations.
% %
% % COMBAK TODO Update number of unique polymorphs discovered to something more meaningful
% % Let's report how many meta-stable IrO2/3 we discovered, much better than how it is currently phrased
% For \IrOtwo, we reaffirm the rutile phase as the globally stable polymorph, while also finding \mytilde200 polymorphs that are within the synthesizability limit.
% %
% For \IrOthree we discovered ~70 unique metastable polymorphs, with \mytilde20 that are more stable than anything previously reported.
% %
% The globally stable polymorph of \IrOthree is a FeF\textsubscript{3} structure type polymorph with a space group number of 167.
% %
% % There are 10 IrO3 polymorphs within 0.2 eV/atom of the lowest structure and 5 IrO3 polymorphs within 0.1 eV/atom of the lowest structure
% % Of these 10, how many were unknown? Maybe 8ish? TODO double check
% %
% %For \IrOthree, we discover 10 previously unknown polymorphs, and the most stable
% %$\alpha$-AlF\textsubscript{3} type and a rutile-like \IrOthree structure with stabilities lower than 0.2 per atom than anything known to date.
% %
% Based on the above results, we constructed a revised bulk Pourbaix diagram of the Ir-H\textsubscript{2}O system by including \IrOthree, and show that our \aIrOthree phase is the dominant and fully stable phase under acidic oxygen evolution reaction (OER) conditions.
% %
% % COMBAK "the OER" or just "OER"
% Using a thermodynamic criteria for the activity towards the OER we find that the above stable \IrOthree polymorphs have significantly lower theoretical overpotentials than the rutile \IrOtwo phase.
% %
% This elevated catalytic activity is due to the weaker and more ideal binding of the OER intermediates on the more oxidized \IrOthree surfaces.
% %
% % TODO Less cheesy closing thoughts
% This work opens up the possibility of drastically accelerating the discovery of novel crystal structure motifs and has implications towards high through-put catalyst discovery efforts.

%__|

