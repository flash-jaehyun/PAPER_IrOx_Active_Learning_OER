%%%%%%%%%%%%%%%%%%%%%%%%%%%%%%%%%%%%%%%%%%%%%%%%%%%%%%%%%%%%%%%%%%%%%%%%%%%%%%%
%%
%%
%%
%%%%%%%%%%%%%%%%%%%%%%%%%%%%%%%%%%%%%%%%%%%%%%%%%%%%%%%%%%%%%%%%%%%%%%%%%%%%%%%


% ################################# Paragraph #################################
Recent advancements in statistical methods, colloquially termed as machine learning,
have revolutionized a tremendous number of fields due to the ease by which we can train models that are flexible enough to regress to data of interest while maintaining predictive power.
Nowhere has this this impact been felt as much as in the field of materials science,
which had previously been bottle-necked by relatively computationally expensive methods.
Herein, we report on a ML methodology to enumerate bulk crystal structures in the \ce{IrO_2} and \ce{IrO_3} space.
