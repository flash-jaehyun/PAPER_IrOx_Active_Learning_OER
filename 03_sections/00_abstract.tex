%%%%%%%%%%%%%%%%%%%%%%%%%%%%%%%%%%%%%%%%%%%%%%%%%%%%%%%%%%%%%%%%%%%%%%%%%%%%%%%
%%
%%
%%
%%%%%%%%%%%%%%%%%%%%%%%%%%%%%%%%%%%%%%%%%%%%%%%%%%%%%%%%%%%%%%%%%%%%%%%%%%%%%%%


% ################################# Paragraph #################################
% %%%%%%%%%%%%%%%%%%%%%%%%%%%%%%%%%%%%%%%%%%%%%%%%%%%%%%%%%%%%%%%%%%%%%%%%%%%%%
% TEMP
% %%%%%%%%%%%%%%%%%%%%%%%%%%%%%%%%%%%%%%%%%%%%%%%%%%%%%%%%%%%%%%%%%%%%%%%%%%%%%
% | - Paragraph start
%
Machine learning (ML) has revolutionized a number of scientific fields where it is possible to train models that are flexible enough to regress to data of interest while maintaining predictive power.
%
Particular impact of ML surrogate models to  have been in the field of materials science,
where the bottle-neck imposed by the computational expense of Density Functional Theory (DFT) when applied to to vast composition-space of bulk systems has been addressed.
%
However, surrogate model applications to the structural space of bulk-crystals, i.e. polymorphs, are very scarce due to {\bf XX}.
%
Herein, we report on an active learining ML methodology to enumerate bulk crystal motiffs of nearly all possible atomic structures of the \ce{IrO_2} and \ce{IrO_3} stochiometries.
%
Our results show that it is possible to enumerate the structural space using experimentally derived prototypes.
%
Next, we demonstrate that the acceleration of the electronic and structural DFT-based optimization with convergence of the active learning loop in less than {\bf XX} steps,
provided that no-large structural drift is observed.
%
For \ce{IrO_2}, we show that while the bulk-rutile system is the globally stable polymorph,
while more than {\bf XX} unique structures is discovered within 0.2 eV per atom.
%
For \ce{IrO_3}, we discover {\bf XX} previously unknown polymorphs, e.g., $\rm \alpha-AlF$_3$ type and rutile-like \ce{IrO_3} with stabilities lower than 0.2 per atom than anything known to date.
% Add in result of new phase of IrO3 that has implications to OER chemistries (@Raul)
With these results as inputs, we construct a new bulk Pourbiax diagram of the Ir-H$_2$O system.
%
We computationally test the proficiency of these phases towards oxygen-evolution reaction and find that the above stable \ce{IrO_3} polymorphs have much higher activity that any \ce{IrO_2}.
%
This work opens up an opportunity to materials/catalysts structural discovery on unprecedented scale.     
% __|
