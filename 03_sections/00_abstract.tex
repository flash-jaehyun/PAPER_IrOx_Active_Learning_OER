%%%%%%%%%%%%%%%%%%%%%%%%%%%%%%%%%%%%%%%%%%%%%%%%%%%%%%%%%%%%%%%%%%%%%%%%%%%%%%%
% Abstract %%%%%%%%%%%%%%%%%%%%%%%%%%%%%%%%%%%%%%%%%%%%%%%%%%%%%%%%%%%%%%%%%%%%
% %%%%%%%%%%%%%%%%%%%%%%%%%%%%%%%%%%%%%%%%%%%%%%%%%%%%%%%%%%%%%%%%%%%%%%%%%%%%%
%
%%%%%%%%%%%%%%%%%%%%%%%%%%%%%%%%%%%%%%%%%%%%%%%%%%%%%%%%%%%%%%%%%%%%%%%%%%%%%%%



%
Machine learning (ML) based surrogate models have become an increasingly common tool to overcome the relative expense of density functional theory in the field of computational materials discovery.
%
The application of ML surrogate models trained on expansive data sets of ab-initio DFT data has been used to discover new materials in the vast composition-space.
%
However, surrogate model applications to the structural space of bulk-crystals, i.e. polymorphs, are very scarce due to the inherent bias of ab-initio datasets towards systems with a large variety in composition but with limited structural diversity.
%
Herein, we report on an active learning ML methodology that searches for the most stable crystal structures of \IrOtwo and \IrOthree by utilizing surrogate models which optimize within a candidate data set of crystal motifs sourced from publicly available materials databases.
% NUMBER
We demonstrate the efficacy of this AL-accelerated methodology by discovering 70 percent of the 10 most stable crystal structures for \IrOtwo and \IrOthree with less than 50 DFT calculations.
% NUMBER
For \IrOtwo, we reaffirm the rutile phase as the globally stable polymorph,
while for the previously unexplored \IrOthree more than {\bf XX} unique structures are discovered within the metastability limit of 0.2 eV per atom.
%
% There are 10 IrO3 polymorphs within 0.2 eV/atom of the lowest structure and 5 IrO3 polymorphs within 0.1 eV/atom of the lowest structure
% Of these 10, how many were unknown? Maybe 8ish? TODO double check
% NUMBER
For \IrOthree, we discover 10 previously unknown polymorphs, e.g.,
$\alpha$-AlF\textsubscript{3} type and a rutile-like \IrOthree structure with stabilities lower than 0.2 per atom than anything known to date.
% Add in result of new phase of IrO3 that has implications to OER chemistries (@Raul)
%
With these results as inputs, we construct a new bulk Pourbaix diagram of the Ir-H\textsubscript{2}O system.
%
We computationally test the proficiency of these phases towards oxygen evolution reaction and find that the above stable \IrOthree polymorphs have much higher activity that any \IrOtwo.
%
This work opens up an opportunity to materials/catalysts structural discovery on unprecedented scale.
