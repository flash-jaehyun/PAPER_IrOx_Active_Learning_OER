%%%%%%%%%%%%%%%%%%%%%%%%%%%%%%%%%%%%%%%%%%%%%%%%%%%%%%%%%%%
% Abstract %%%%%%%%%%%%%%%%%%%%%%%%%%%%%%%%%%%%%%%%%%%%%%%%
% %%%%%%%%%%%%%%%%%%%%%%%%%%%%%%%%%%%%%%%%%%%%%%%%%%%%%%%%%
%
%%%%%%%%%%%%%%%%%%%%%%%%%%%%%%%%%%%%%%%%%%%%%%%%%%%%%%%%%%%



\noindent The discovery of high-performing and stable materials for sustainable energy applications is a pressing goal in catalysis and materials science.
%
Understanding the relationship between a material's structure and functionality is an important step in the process, such that viable polymorphs for a given chemical composition need to be identified.
%
Machine-learning based surrogate models have the potential to accelerate the search for polymorphs that target specific applications.
%
Herein, we report a readily generalizable active-learning (AL) accelerated algorithm for identification of electrochemically stable iridium-oxide polymorphs of \IrOtwo and \IrOthree.
%
The search is coupled to a subsequent analysis of the electrochemical stability of the discovered structures for the acidic oxygen evolution reaction (OER).
%
Structural candidates are generated by identifying all \num{956} structurally unique \ABtwo and \ABthree prototypes in existing materials databases (more than \num{38000}).
%
Next, using an active learning approach we are able to find \num{196} \IrOtwo polymorphs within the thermodynamic amorphous synthesizability limit and reaffirm the global stability of the rutile structure.
%
We find \num{75} synthesizable \IrOthree polymorphs and report a previously unknown \ce{FeF_{3}}-type structure as the most stable, termed \aIrOthree.
%
To test the algorithms performance,
we compare to a random search of the candidate space and report at least a twofold increase in the rate of discovery.
%
Additionally, the AL approach can acquire the most stable polymorphs of \IrOtwo and \IrOthree with less than \num{30} density functinoal theory optimizations.
%
Analysis of the structural properties of the discovered polymorphs reveals that octahedral local coordination environments are preferred for nearly all low energy structures.
%
Subsequent Pourbaix Ir-H$_2$O analysis shows that \aIrOthree is the globally stable solid phase under acidic OER conditions and supersedes the stability of rutile \IrOtwo.
%
Calculation of theoretical OER surface activities reveal ideal weaker binding of the OER intermediates on \aIrOthree than on any other considered iridium-oxide.
%
We emphasize that the proposed AL algorithm can be easily generalized to search for any binary metal-oxide structure with a defined stoichiometry.
