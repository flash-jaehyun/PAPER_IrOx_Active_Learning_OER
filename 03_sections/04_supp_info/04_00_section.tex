% ###########################################################################

% | - Machine Learning Algorithm Methods
\subsection{Machine Learning Algorithm Methods}

Relevant details about the ML Gaussian process here % @Chris
% __|

% | - Electrochemical OER Computational Methods
\subsection{Electrochemical OER Computational Methods}

% | - Density Functional Theory Methods
\subsubsection{Density Functional Theory Methods}
% Get VASP #reference in the ex. word doc. that @Michal shared with me
All OER calculations were performed using density functional theory (DFT) implemented via the Vienna ab-initio simulation package (VASP) and utilizing the PBE exchange-correlation functional.
Dipole corrections were imposed on all non-symmetric slabs.
A 4x4x3 k-point mesh with gamma-point centered Monkshort-packing was used for all slabs.
The plane-wave energy cutoff was 500 eV.

% COMBAK Figure out how much spacing was used for all slabs
% FIXME Change the A -> Angstrom symbol

% COMBAK What kind of optimization routine was used? (Newtonian, BFGS?)
All slab calculations maintained a vacuum spacing of <15 A.
% COMBAK change A -> Angstrom
All structures were relaxed utilizing a TEMP algorithm with a stop criteria being that all atoms satisfy a maximum force threshold of 0.02 eV/A.
% __|

% | - OER Thermodynamic Methodology
\subsubsection{OER Thermodynamic Methodology}
% __|

% | - Surface Energy Pourbaix Methodology
\subsubsection{Surface Energy Pourbaix Methodology}
% __|

Procedure:
- For the top/most stable bulk structures the following procedure was carried out

* Stable stoicheometric terminations were cut from the bulk
    Stable termination planes were guesstimated via intuition, and the x-ray diffraction pattern tool from Vesta

* Electrochemical surface coverage was elucidated via a surface Pourbaix analysis
    Need to know the coverage of surface under operating conditions (>1.23 V RHE)

* Thermodynamic/limiting potential analysis of the OER mechanistic pathway
    Volcano plot, limiting potentials, etc.

% TODO Gibbs corrections, OER mechanism
% __|
