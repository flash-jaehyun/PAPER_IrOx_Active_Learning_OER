% %%%%%%%%%%%%%%%%%%%%%%%%%%%%%%%%%%%%%%%%%%%%%%%%%%%%%%%%%
% | - Conclusions
%
% Final conclusions and parting thoughts
%
% __|
% %%%%%%%%%%%%%%%%%%%%%%%%%%%%%%%%%%%%%%%%%%%%%%%%%%%%%%%%%


% %%%%%%%%%%%%%%%%%%%%%%%%%%%%%%%%%%%%%%%%%%%%%%%%%%%%%%%%%
% | - PARAGRAPH HEADER
% Conclusions on AL and results
% __|
% %%%%%%%%%%%%%%%%%%%%%%%%%%%%%%%%%%%%%%%%%%%%%%%%%%%%%%%%%
% | - PARAGRAPH BODY
%
% \textbf{Generation of structurally diverse search space and efficient sampling of it.}
We have described a cogent procedure for generating and searching a structurally diverse candidate space of bulk structural prototypes with a desired composition.
%
Once this space is enumerated, we show how it can be efficiently searched using an algorithm with an active learning loop without a prior knowledge of accurate atomic positions.
%
In most cases, the DFT optimization of only a fraction of the candidates leads to identification of the most stable polymorphs.
%
In particular, this approach is well-suited for discovery in structurally diverse structures, such as metal oxides and other metal-ligand bulk systems, where there exits a large degree of structural diversity.
%
The current dataset includes octahedral, tetrahedral, square-pyramidal, cubic, and square-planar Ir-O conformers.
%
We also note, that our AL algorithm is capable of discovering experimentally known phases such as pyrite, columbite and layered \IrOtwo and several recently discovered layered \IrOthree phases formed by Li$^+$ deintercalation.
%
In particular, we have identified a number of previously unknown \IrOthree polymorphs below the amorphous synthesizability limit,
including a new globally stable \aIrOthree phase.
%
This high valency Ir$^{6+}$ phase is stable under OER relevant conditions and has an ideal 100\% corner-sharing octahedral structure, a short Ir-O bond length of 1.93 \angstrom, and also has a very high surface coverage of active oxygens.
%
Calculations of surface thermodynamics reveal this structure and other OER stable \IrOthree phases have much higher theoretical OER activity than a benchmark rutile \IrOtwo.
%
The thermodynamic stability and high OER activity of the \aIrOthree phase may provide clues as to the nature of the yet uncharacterized structures reported after reconstruction of \ce{SrIrO_{3}} and \IrOx precursors under OER reaction conditions.
%
Methods combining diverse structural generation, AL-enabled accelerated searches, and \mbox{ab-initio} simulation of material performance could open up new avenues for in silico material design with application tailored structural properties.
% __|
% %%%%%%%%%%%%%%%%%%%%%%%%%%%%%%%%%%%%%%%%%%%%%%%%%%%%%%%%%
