%%%%%%%%%%%%%%%%%%%%%%%%%%%%%%%%%%%%%%%%%%%%%%%%%%%%%%%%%%%%%%%%%%%%%%%%%%%%%%%
%% Conclusions
%%
%%
%%%%%%%%%%%%%%%%%%%%%%%%%%%%%%%%%%%%%%%%%%%%%%%%%%%%%%%%%%%%%%%%%%%%%%%%%%%%%%%



% ################################# Paragraph #################################
% %%%%%%%%%%%%%%%%%%%%%%%%%%%%%%%%%%%%%%%%%%%%%%%%%%%%%%%%%%%%%%%%%%%%%%%%%%%%%
% Conclusions on ALL and results
% %%%%%%%%%%%%%%%%%%%%%%%%%%%%%%%%%%%%%%%%%%%%%%%%%%%%%%%%%%%%%%%%%%%%%%%%%%%%%
% | - Paragraph start
%
In conclusion, we have demonstrated an active-learning accelerated algorithm for the discovery of stable crystal polymorphs by searching through a candidate space of structurally distinct iridium-oxide phases.
%
% TODO Can I generalize this result for both IrO2 and IrO3?
% TODO Use percentage of total systems instead of just # of DFT calcs
The algorithm can identify \num{7} of the \num{10} most stable polymorphs of \IrOthree with only \num{35} DFT bulk relaxations and TEMP of the most \num{10} stable \IrOtwo polymorphs after TEMP DFT calculations.
%
For \IrOtwo, we find....  For \IrOthree we find....
%
% Implications for for future studies of polymorphs
In the \IrOtwo space our search failed to uncover anything more stable than the \rIrOtwo phase,
while for \IrOthree (a much less explored stoichiometry) we found several polymorphs phases that are predicted to be stable under OER conditions.
%
% Describe the level of acceleration achieved, effect of structural drift, implications for future studies
We have analyzed the local and global structural coordination and revealed a large degree of structural diversity in our dataset (octahedral, tetrahedral, square-pyramidal, cubic, and square-planar)
%
% Discuss the differences  connectivity/M-O bonds lengths is different between IrO2 and IrO3
Although octahedral coordination environments are energetically preferred TEMP TEMP.
%
The most stable systems were used to construct a revised Pourbaix diagram of Ir-H$_2$O system.
%
Very importantly, we predict that \IrOthree is the thermodynamically preferred phase under OER conditions.
% __|


% ################################# Paragraph #################################
% %%%%%%%%%%%%%%%%%%%%%%%%%%%%%%%%%%%%%%%%%%%%%%%%%%%%%%%%%%%%%%%%%%%%%%%%%%%%%
% Conclusions OER results (APPLICATIONS)
% %%%%%%%%%%%%%%%%%%%%%%%%%%%%%%%%%%%%%%%%%%%%%%%%%%%%%%%%%%%%%%%%%%%%%%%%%%%%%
% | - Paragraph start
%
Finally, using a thermodynamic approach to OER, we show,
that surfaces of selected \IrOthree have much higher relative activity that \IrOtwo due to presence of high valency Ir[6+] states.
%
The \num{100}\% corner-sharing octahedral structures feature maximum coverage of oxygens with optimal $2p$-energy.
%
The OER results has broader implications for related OER systems.
%
Overall, the AL ML has tremendous potential for discovery of structurally diverse systems particularly where the known diversity is currently very small (Highly oxidized oxides).
%
Because our method provides readily available structural information which a necessary input for any characterization/simulation analysis TEMP We envision that .....
%
This opens up and new avenues of the materials/catalysis research with tailored structural properties.
% __|


% ################################# Paragraph #################################
% %%%%%%%%%%%%%%%%%%%%%%%%%%%%%%%%%%%%%%%%%%%%%%%%%%%%%%%%%%%%%%%%%%%%%%%%%%%%%
% Future works paragraph
% %%%%%%%%%%%%%%%%%%%%%%%%%%%%%%%%%%%%%%%%%%%%%%%%%%%%%%%%%%%%%%%%%%%%%%%%%%%%%
% | - Paragraph start
Going forward we will improve


% __|
