%%%%%%%%%%%%%%%%%%%%%%%%%%%%%%%%%%%%%%%%%%%%%%%%%%%%%%%%%%%
% | - Conclusions %%%%%%%%%%%%%%%%%%%%%%%%%%%%%%%%%%%%%%%%%
% TEMP
% __| %%%%%%%%%%%%%%%%%%%%%%%%%%%%%%%%%%%%%%%%%%%%%%%%%%%%%



% ####################### Paragraph #######################
% %%%%%%%%%%%%%%%%%%%%%%%%%%%%%%%%%%%%%%%%%%%%%%%%%%%%%%%%%
% Conclusions on AL and results
% | - %%%%%%%%%%%%%%%%%%%%%%%%%%%%%%%%%%%%%%%%%%%%%%%%%%%%%
%
In conclusion, we have demonstrated an active-learning accelerated algorithm for the discovery of stable crystal polymorphs by searching through a candidate space of structurally distinct iridium-oxide phases.
%
% TODO Can I generalize this result for both IrO2 and IrO3?
% TODO Use percentage of total systems instead of just # of DFT calcs
The algorithm can identify a majority of the most stable polymorphs (7 of the 10 most stable) in a candidate set after only computing a fraction of them via DFT (\mytilde90 DFT optimizations).
%
For \IrOtwo, we find confirm the rutile phase as the most stable crystal structure, while also finding several well known phases, including anatase, columbite, as well as several new phases of \IrOtwo.
%
For the relatively unexplored \IrOthree we found a new globally stable phase (\aIrOthree), a completely corner sharing octahedral structure with space group 182.
%
% Implications for for future studies of polymorphs
%
% Describe the level of acceleration achieved, effect of structural drift, implications for future studies
We have analyzed the local and global structural coordination and revealed that octahedral coordination environments are predominitly preferred, although we also have a large degree of structural diversity in our dataset (octahedral, tetrahedral, square-pyramidal, cubic, and square-planar are all represented)
%
% Discuss the differences  connectivity/M-O bonds lengths is different between IrO2 and IrO3
%
We constructed a revised bulk Pourbaix diagram for Ir-H$_2$O, including the newly found \aIrOthree phase and revealed that \aIrOthree has a substantial window of stability in the OER relevant potentials and pH.
% __| %%%%%%%%%%%%%%%%%%%%%%%%%%%%%%%%%%%%%%%%%%%%%%%%%%%%%

% ####################### Paragraph #######################
% %%%%%%%%%%%%%%%%%%%%%%%%%%%%%%%%%%%%%%%%%%%%%%%%%%%%%%%%%
% Conclusions OER results (APPLICATIONS)
% | - %%%%%%%%%%%%%%%%%%%%%%%%%%%%%%%%%%%%%%%%%%%%%%%%%%%%%
%
Finally, using a thermodynamic approach to OER, we show,
that surfaces of selected \IrOthree have much higher relative activity that \IrOtwo due to presence of high valency Ir[6+] states.
%
The \num{100}\% corner-sharing octahedral structures feature maximum coverage of oxygens with optimal $2p$-energy.
%
The OER results has broader implications for related OER systems.
%
Overall, the AL ML has tremendous potential for discovery of structurally diverse systems particularly where the known diversity is currently very small (Highly oxidized oxides).
%
Because our method provides readily available structural information which a necessary input for any characterization/simulation analysis we envision it as a indispensable tool in identifying novel phases for materials applications.
%
This opens up new avenues of the materials/catalysis research with tailored structural properties.
% __| %%%%%%%%%%%%%%%%%%%%%%%%%%%%%%%%%%%%%%%%%%%%%%%%%%%%%

% ####################### Paragraph #######################
% %%%%%%%%%%%%%%%%%%%%%%%%%%%%%%%%%%%%%%%%%%%%%%%%%%%%%%%%%
% Future works paragraph
% | - %%%%%%%%%%%%%%%%%%%%%%%%%%%%%%%%%%%%%%%%%%%%%%%%%%%%%
% Going forward we will improve

% __| %%%%%%%%%%%%%%%%%%%%%%%%%%%%%%%%%%%%%%%%%%%%%%%%%%%%%
