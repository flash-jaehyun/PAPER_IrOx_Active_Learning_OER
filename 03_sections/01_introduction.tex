% %%%%%%%%%%%%%%%%%%%%%%%%%%%%%%%%%%%%%%%%%%%%%%%%%%%%%%%%%
% | - Introduction
% %%%%%%%%%%%%%%%%%%%%%%%%%%%%%%%%%%%%%%%%%%%%%%%%%%%%%%%%%
%
% Key points:
%   * Materials are composed of their "stoichiometry" and their "structure"
%   * 3N dimensional Cartesian space of coordinates is prohibitively large
%   * Our candidate space generation leverages symmetry to produce a set of structurally unique and well distributed (across the high dim. 3N space) polymorphs
%   * Mention other methods for bulk crystal structure Discovery
    %     - simulated annealing
%     - paper that Hammer put out (R58)
% __|
%%%%%%%%%%%%%%%%%%%%%%%%%%%%%%%%%%%%%%%%%%%%%%%%%%%%%%%%%%%



% %%%%%%%%%%%%%%%%%%%%%%%%%%%%%%%%%%%%%%%%%%%%%%%%%%%%%%%%%
% | - High Level Intro to Problem
% * General intro into problem, give proper context
% * Make the point that structure has been all but ignored in recent years by the big-data/ML material science community (OQMD MP as examples)
%
% OUTLINE:
%  * Solving for crystal structures in a generally difficult problem
%  * It's an important problem to solve because all chemical/physical properties of a material are fundamentally linked to the atomic structure.
%  * Although we've seen a lot of work in the past ~10 years from material science groups leveraging recent advances in big-data, ML, cheaper computational resources, high-throughput screening, materials databases, etc.
%    * Most of the focus has been on exploring compositions and not structure
% __|
% %%%%%%%%%%%%%%%%%%%%%%%%%%%%%%%%%%%%%%%%%%%%%%%%%%%%%%%%%
% | - PARAGRAPH BODY
%
% TODO this intro sentence will be a reference dump for literature in crystal structure prediction
To understand or simulate the properties of novel polymorphs of functional materials their crystal structure must first be solved for,
which remains a challenging problem in materials science.
\cite{Woodley2008,Graser2018}
%
Large high-throughput \textit{ab initio} datasets
\cite{Saal2013,Jain2013,Curtarolo2012, mamun2019high}
have enabled approaching many problems in materials research with machine-learning,
\cite{Kirklin2015}
but these datasets are systematically biased towards known materials or hypothetical materials derived from common crystal prototypes.
%
Thus, there is a need for the systematic exploration of structural diversity at target elemental compositions.
%to be able to quantitatively relate computational predictions with experimental measurements in materials applications.
% __|
% %%%%%%%%%%%%%%%%%%%%%%%%%%%%%%%%%%%%%%%%%%%%%%%%%%%%%%%%%


% %%%%%%%%%%%%%%%%%%%%%%%%%%%%%%%%%%%%%%%%%%%%%%%%%%%%%%%%%
% | - Introduction to GO and Crystal Structure Predication
% * Arguments and motivations for our unique approach
%
% NOTES:
%   * Emphasize the candidate space generation as a KEY feature of algorithm
%
% __|
% %%%%%%%%%%%%%%%%%%%%%%%%%%%%%%%%%%%%%%%%%%%%%%%%%%%%%%%%%
% | - PARAGRAPH BODY
%
% TODO All global optimization papers go here
% TODO Add review paper for GO
Contemporary approaches to inorganic crystal structure prediction %involve a variety of different strategies of exploring the high-dimensional potential energy landscape to find local or global minima,
%with methods ranging from
include simulated annealing, evolutionary algorithms, and particle swarm optimization.
\cite{Oganov2006,Lyakhov2010,Glass2006,Wang2012,Pickard2011,Pickard2006,Avery2019,Lonie2011}
%
% TODO Find solid example of shortcomings of classical GO schemes
These approaches are comprehensive, but become intractable as  %provide comprehensive,
%and often successful searches for stable polymorphs but tend to hit a tractability wall with
%increasing system size as the number of
polymorphic configurations increase exponentially with the number and types of elements considered~\cite{Stillinger1999}.
%
%Thus, there is a need for
%Generalizable crystal structure prediction algorithms with low computational cost.% and are generalizable. as previously reported methods while reducing the computational cost associated with uncovering new novel polymorphic phases.
% __|
% %%%%%%%%%%%%%%%%%%%%%%%%%%%%%%%%%%%%%%%%%%%%%%%%%%%%%%%%%
% %%%%%%%%%%%%%%%%%%%%%%%%%%%%%%%%%%%%%%%%%%%%%%%%%%%%%%%%%
% | - Introduce Active Learning Papers
%
% __|
% %%%%%%%%%%%%%%%%%%%%%%%%%%%%%%%%%%%%%%%%%%%%%%%%%%%%%%%%%
% | - PARAGRAPH BODY
%
% COMBAK NOTE \textbf{These references don't seem very balanced. I'm sure there are other papers outside of the Norskov group we can cite here}
% TODO Find more papers of AL in materials pace to put here
% TODO Find references from Logan Ward's and Turab Lookman
Recent materials discovery approaches employing surrogate models in lieu of Density Functional Theory (DFT) calculations include a tight-binding model utilizing genetic algorithms~\cite{VanDenBossche2018},
agent-based rapid generation of phase diagrams in diverse chemistries~\cite{Montoya2020},
and an image-based materials representation procedure from Noh \latin{et al.}~\cite{noh2019inverse},
which was used to find stable vanadium oxide polymorphs.
%
Active learning (AL) frameworks in conjunction with surrogate models have emerged as a computationally efficient approach for problems in materials science, and a potential alternative to established crystal structure prediction (CSP) methods.
%an effective approach to successfully speed up various problems in materials science,
%and hence have the potential to deliver an efficient alternative and complimentary approach to established crystal structure prediction (CSP) methods.
\cite{hansen2019atomistic,torres2019low,Jennings2019,podryabinkin2017active,Bassman2018}
%
% This sentence will be a dump for other recent GO papers that aren't necessarily just AL

% __|
% %%%%%%%%%%%%%%%%%%%%%%%%%%%%%%%%%%%%%%%%%%%%%%%%%%%%%%%%%


% %%%%%%%%%%%%%%%%%%%%%%%%%%%%%%%%%%%%%%%%%%%%%%%%%%%%%%%%%
% | - Introduce IrOx Chemical Space
% Introduce specific system (Ir-oxides), relevance (OER), and what is to be gained by discovering new polymorphs
% __|
% %%%%%%%%%%%%%%%%%%%%%%%%%%%%%%%%%%%%%%%%%%%%%%%%%%%%%%%%%
% | - PARAGRAPH BODY
%
In this manuscript we report a rapid crystal structure discovery approach that leverages machine learning surrogate models and an AL framework to accelerate the discovery of polymorphs at target chemical compositions.
% ,and show specific applications in two important \IrOx compositions (x=2 and 3) relevant for electrocatalyst applications.
%
Our method does not rely on the existence of past DFT data,
but instead sequentially generates the minimum sized data set to effectively search within a structurally diverse space of candidates generated from crystal structure prototypes.
%
%The algorithm avoids operating in the structural space by leveraging nature's propensity for symmetry by preparing data sets with a large degree of structural diversity at fixed composition.
% this sounds little hand waving | RF Maybe it needs more exposition, I'm convoluting 2 points here, that we don't do traditional GO opt. in cont. PES and that we use the intuition of symmetry to our advantage
%
%As an alternative approach to GO,
% We first define a set of structurally diverse candidate crystal polymorphs, and secondly, search through the static list of candidates with a selection-type algorithm.
%
%This approach relies on being able to prepare candidate structures that are physically reasonable and likely low in energy.
%
%To date, there have been various techniques to prepare candidate structures, including generating structures randomly, enumerating space groups, etc. \textbf{Refs?}
%
% This looks out of space. Consider moving up or eliminate. Also, this is more a discussion statement.
%
% Empirically, we know that nature tends to favor symmetric structures, and so we will leverage this intuition by building a dataset explicitly enumerating through space groups.
%
% TODO COMBAK Battery applications?
We demonstrate the application of this methodology in the space of iridium oxide polymorphs,
an important class of materials with applications in electrochemistry,
but with unresolved structure-activity properties critical for understanding their catalytic activity.
%
% See following link for list of important IrO2 OER papers:
% https://workflowy.com/#/e755141e2f66
In particular, Rutile-\ce{IrO_2} (\rIrOtwo) (\ce{Ir^{4+}}), the most stable form of iridium-oxide at standard conditions,
is a well-studied electrocatalyst for the oxygen evolution reaction (OER).
\cite{Seitz2016,Lee2012a,McCrory2015,Trotochaud2012,Danilovic2014,Carmo2013,Miles1978,Beni1979}
%
% \textbf{Need refer to the Schloegl work in this section}
% Which Schloegl papers are you all referring to?
Previous studies on a SrIrO\textsubscript{3} OER electrocatalyst demonstrated that Sr leaching might leave behind a highly oxidized \ce{Ir^{6+}} species which was argued to be responsible for the observed OER activity.
\cite{Seitz2016}
%
% @Michal wants a Schlogl paper here, not sure which TODO
Other groups also observed reconstruction of \IrOx catalysts under reaction conditions and subsequent formation of an unknown structure.~\cite{Pearce2017}
%
% COMBAK More references here (Chris)
Highly oxidized \IrOthree phases are also formed as the terminal structure of Li\textsubscript{x}IrO\textsubscript{3} anodes.\cite{Pearce2017}
%
% TODO Find reference for IrOxHy phases and their importance
For these reasons, we focused our search on stable polymorphs in the \IrOtwo stoichiometry and oxidized \IrOthree stoichiometry.
%
% TODO Add references to octahedra literature
%Furthermore, such pure corner sharing octahedral crystals are known from in other systems such fluorites and chlorites.
% __|
% %%%%%%%%%%%%%%%%%%%%%%%%%%%%%%%%%%%%%%%%%%%%%%%%%%%%%%%%%


% %%%%%%%%%%%%%%%%%%%%%%%%%%%%%%%%%%%%%%%%%%%%%%%%%%%%%%%%%
% | - Outlining the Structure of the Paper
%
% __|
% %%%%%%%%%%%%%%%%%%%%%%%%%%%%%%%%%%%%%%%%%%%%%%%%%%%%%%%%%
% | - PARAGRAPH BODY
%
Here, we first detail the generation of our candidate structures for \IrOtwo and \IrOthree and introduce the AL accelerated surrogate model.
%
Next, we demonstrate the application of our AL scheme to the \IrOtwo and \IrOthree prototype spaces and evaluate model performance for acquisition of the most stable polymorphs.
%show the performance of our routine at finding the most stable polymorphs in the candidate sets.
%
We analyze the crystallographic motifs of the DFT-relaxed structures and identify structural trends within the most stable polymorphs.
%
Lastly, we incorporate discovered structures into bulk and surface Pourbaix diagrams, and evaluate their catalytic OER performance.% for the most stable phases.
%
%highlighting the importance of the \IrOthree phases under OER.
% __| %%%%%%%%%%%%%%%%%%%%%%%%%%%%%%%%%%%%%%%%%%%%%%%%%%%%%
% %%%%%%%%%%%%%%%%%%%%%%%%%%%%%%%%%%%%%%%%%%%%%%%%%%%%%%%%%
