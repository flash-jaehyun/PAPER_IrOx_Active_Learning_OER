%%%%%%%%%%%%%%%%%%%%%%%%%%%%%%%%%%%%%%%%%%%%%%%%%%%%%%%%%%%
% | - Introduction %%%%%%%%%%%%%%%%%%%%%%%%%%%%%%%%%%%%%%%%
% %%%%%%%%%%%%%%%%%%%%%%%%%%%%%%%%%%%%%%%%%%%%%%%%%%%%%%%%%
%
% Key points:
%   * Materials are composed of their "stoichiometry" and their "structure"
%   * 3N dimensional Cartesian space of coordinates is prohibitively large
%   * Our candidate space generation leverages symmetry to produce a set of structurally unique and well distributed (across the high dim. 3N space) polymorphs
%   * Mention other methods for bulk crystal structure Discovery
    %     - simulated annealing
%     - paper that Hammer put out (R58)
    % __| %%%%%%%%%%%%%%%%%%%%%%%%%%%%%%%%%%%%%%%%%%%%%%%%%%%%%



% %%%%%%%%%%%%%%%%%%%%%%%%%%%%%%%%%%%%%%%%%%%%%%%%%%%%%%%%%
% ####################### Paragraph #######################
% %%%%%%%%%%%%%%%%%%%%%%%%%%%%%%%%%%%%%%%%%%%%%%%%%%%%%%%%%
% * General intro into problem, give proper context
% * Make the point that structure has been all but ignored in recent years by the big-data/ML material science community (OQMD MP as examples)
%
% OUTLINE:
%  * Solving for crystal structures in a generally difficult problem
%  * It's an important problem to solve because all chemical/physical properties of a material are fundamentally linked to the atomic structure.
%  * Although we've seen a lot of work in the past ~10 years from material science groups leveraging recent advances in big-data, ML, cheaper computational resources, high-throughput screening, materials databases, etc.
%    * Most of the focus has been on exploring compositions and not structure
% %%%%%%%%%%%%%%%%%%%%%%%%%%%%%%%%%%%%%%%%%%%%%%%%%%%%%%%%%
% | - %%%%%%%%%%%%%%%%%%%%%%%%%%%%%%%%%%%%%%%%%%%%%%%%%%%%%
% TODO this intro sentence will be a reference dump for literature in crystal structure prediction
Predicting the thermodynamically favorable crystal structures for an arbitrary inorganic system remains a challenging problem in computational material science.\cite{Woodley2008}
%
% Experimentally synthesized inorganic materials often conform to the global minimum energy structure (or similarly stable meta-stable phases).
%
% TODO Find reference to backup the importance of knowing the crystal structure to predict properties of real-world systems. I would consider this to be common knowledge but relevant article would be nice.
To simulate the material properties of new meta-stable polymorphs, which are close in energy to the convex hull of stability, their structures must first be found.
%When simulations are used to guide the search for new materials, the stable and meta-stable crystal structures, \latin{i.e.} polymorphs,  close to the convex hull of stability must be known in order to predict the material properties.\textbf{the previous sentence doesn't make sense, I'm not sure what you are trying to say here}
%
% Structure vs. composition enumeration (composition done to death)
% Give example of how over represented some motifs are
%
% TODO Cite multiple materials databases here
Although there have been numerous examples in recent years of machine learning algorithms applied towards the prediction of formation energies of large \latin{ab-initio} data sets,
these data sets are biased towards common structures and varying composition space.
%
% TODO Input the number or fraction of AB3 Perokskite structures that are present in OQMD
For example the Open Quantum Materials Database (OQMD) contains a large dataset of ternary-alloy combinations in the same close-packed cubic structure.\cite{Kirklin2015}
%
% Should we say that enumerating comp. is easy and struct. is hard?
As such, these efforts have been primarily concerned with the enumeration of composition (elemental identity and stoichiometries) and less so with the exploration of structural diversity.
%
% There are way more papers in the general area of GO than these
% I'm leaning away from framing as a "structural-diversity" problem with the databases and more so as a method of global optimization
% In the past, various groups have put forth methodologies to address the structural-diversity problem using computation for the MnO2~ and VOx~ polymorph spaces,
% but most of these methodologies are limited by the fact that they are operating within the highly intractable/dimensional PES.
%
% __| %%%%%%%%%%%%%%%%%%%%%%%%%%%%%%%%%%%%%%%%%%%%%%%%%%%%%

% %%%%%%%%%%%%%%%%%%%%%%%%%%%%%%%%%%%%%%%%%%%%%%%%%%%%%%%%%
% ####################### Paragraph #######################
% %%%%%%%%%%%%%%%%%%%%%%%%%%%%%%%%%%%%%%%%%%%%%%%%%%%%%%%%%
% * Introduction to global optimization and crystal structure predication community
% * Arguments and motivations for our unique approach
%
% NOTES:
%   * Emphasize the candidate space generation as a KEY feature of algorithm
%
% %%%%%%%%%%%%%%%%%%%%%%%%%%%%%%%%%%%%%%%%%%%%%%%%%%%%%%%%%
% | - %%%%%%%%%%%%%%%%%%%%%%%%%%%%%%%%%%%%%%%%%%%%%%%%%%%%%
% TODO All global optimization papers go here
% TODO Add review paper for GO
% Highly dimensional is also next sentence, but this sentence is a good spot for it though
Historically, the most common approach to crystal structure prediction relied on classical global optimization schemes,
in which the local and global minimum are found via numerical optimization routines that operate within the continuous potential energy surface (PES).
%
% TODO Find solid example of shortcomings of classical GO schemes
These approaches are only tractable for simple systems, such as close-packed metallic crystals,
as the number of potential polymorphs increases exponentially with systems size and degrees of freedom.\cite{Stillinger1999}
%
The class of structurally diverse metal-oxides are a prime example of a chemical space for which structure prediction is non-trivial, as materials tend to organize into a variety of well-defined local coordination environments such octahedra, tetrahedra.
%
These local coordination motifs can further assemble to form systems with long-range order.
\textbf{Trying to say that the infinite number of ways to pack octahedra lead to a very very large number of possible polymorphs}
% __| %%%%%%%%%%%%%%%%%%%%%%%%%%%%%%%%%%%%%%%%%%%%%%%%%%%%%

% %%%%%%%%%%%%%%%%%%%%%%%%%%%%%%%%%%%%%%%%%%%%%%%%%%%%%%%%%
% ####################### Paragraph #######################
% %%%%%%%%%%%%%%%%%%%%%%%%%%%%%%%%%%%%%%%%%%%%%%%%%%%%%%%%%
% Start talking about AL background
% %%%%%%%%%%%%%%%%%%%%%%%%%%%%%%%%%%%%%%%%%%%%%%%%%%%%%%%%%
% | - %%%%%%%%%%%%%%%%%%%%%%%%%%%%%%%%%%%%%%%%%%%%%%%%%%%%%
%
% COMBAK NOTE \textbf{These references don't seem very balanced. I'm sure there are other papers outside of the Norskov group we can cite here}
% TODO Find more papers of AL in materials pace to put here
Alternatively, active learning (AL) frameworks utilizing surrogate models have been demonstrated to successfully speed up \latin{ab initio} calculations, including structural optimizations\cite{hansen2019atomistic}, transition-state searches \cite{torres2019low}, optimal alloy nano-particle compositions~\cite{Jennings2019} and fitting of inter-atomic potentials.\cite{podryabinkin2017active}
%
% This sentence will be a dump for other recent GO papers that aren't necessarily just AL
There have been numerous recent efforts in developing unique approaches to materials discovery, including a tight-binding model utilizing genetic algorithms~\cite{VanDenBossche2018}, and an image-based materials representation procedure from Noh \latin{et al.}~\cite{noh2019inverse}, which was used to find stable vanadium oxide polymorphs.
%
% To address the limitations of traditional global optimization methods,
%
Here we report on a crystal structure discovery algorithm that leverages machine learning surrogate models and an active learning framework to accelerate the discovery of novel crystal structures at fixed composition and does not rely on large sets of generated DFT data.
%
%The algorithm avoids operating in the structural space by leveraging nature's propensity for symmetry by preparing data sets with a large degree of structural diversity at fixed composition.
% this sounds little hand waving | RF Maybe it needs more exposition, I'm convoluting 2 points here, that we don't do traditional GO opt. in cont. PES and that we use the intuition of symmetry to our advantage
%
%As an alternative approach to GO,
We first define a set of structurally diverse candidate crystal polymorphs, and secondly, search through the static list of candidates with a selection-type algorithm.
%
%This approach relies on being able to prepare candidate structures that are physically reasonable and likely low in energy.
%
% TODO Find references for various ways of preparing candidate structures
%To date, there have been various techniques to prepare candidate structures, including generating structures randomly, enumerating space groups, etc. \textbf{Refs?}
%
% This looks  out of space. Consider moving up or eliminate. Also, this is more a discussion statement.
%
% Empirically, we know that nature tends to favor symmetric structures, and so we will leverage this intuition by building a dataset explicitly enumerating through space groups.
% __| %%%%%%%%%%%%%%%%%%%%%%%%%%%%%%%%%%%%%%%%%%%%%%%%%%%%%

% %%%%%%%%%%%%%%%%%%%%%%%%%%%%%%%%%%%%%%%%%%%%%%%%%%%%%%%%%
% ####################### Paragraph #######################
% %%%%%%%%%%%%%%%%%%%%%%%%%%%%%%%%%%%%%%%%%%%%%%%%%%%%%%%%%
% Introduce specific system (Ir-oxides), relevance (OER), and what is to be
% gained by discovering new polymorphs
% %%%%%%%%%%%%%%%%%%%%%%%%%%%%%%%%%%%%%%%%%%%%%%%%%%%%%%%%%
% | - %%%%%%%%%%%%%%%%%%%%%%%%%%%%%%%%%%%%%%%%%%%%%%%%%%%%%%%%%
We demonstrate this methodology in the space of iridium oxide polymorphs,
an important class of materials with applications in electrochemistry.
%
% See following link for list of important IrO2 OER papers:
% https://workflowy.com/#/e755141e2f66
In particular, \rIrOtwo (Ir[4+] oxidation state), is the most stable form of iridium-oxide at standard conditions,
and is a well studied electrocatalyst for the oxygen evolution reaction (OER).
\cite{Seitz2016,Lee2012a,McCrory2015,Trotochaud2012,Danilovic2014,Carmo2013,Miles1978,Beni1979}
%
% \textbf{Need refer to the Schloegl work in this section}
% Which Schloegl papers are you all referring to?
Previous studies on SrIrO\textsubscript{3} electrocatalyst for the OER demonstrated that Sr leaching might leave behind a highly oxidized Ir (Ir[6+] for hypothetical \IrOthree) and it was argued as one possibly for observed high OER activity.\cite{Seitz2016}
%
% @Michal wants a Schlogl paper here, not sure which TODO
Other groups also observed reconstruction of \IrOx catalysts under reaction conditions and subsequent formation of an unknown structure. \cite{Pearce2017}
%
% COMBAK More references here (Chris)
Highly oxidized \IrOthree phases are also formed as the terminal structure of Li\textsubscript{x}IrO\textsubscript{3} anodes.\cite{Pearce2017}
%
% TODO Find reference for IrOxHy phases and their importance
For these reasons, we focused our study to search for stable polymorphs in the \IrOtwo stoichiometry and  oxidized \IrOthree stoichiometry.
%thereby neglecting the possibility of metal-hydroxide phases which have previously been shown to important.
%
%Purely octahedral \IrOthree leads naturally to 100\% corner sharing octahedra,
%where all terminal surface Ir-oxygens are potentially OER active sites.
%
% TODO Add references to octahedra literature
%Furthermore, such pure corner sharing octahedral crystals are known from in other systems such fluorites and chlorites.


% | - __old__
%Reported +6 oxidation state phases are achievable leading to high degree of structural variability, which is the highest for transition metals.
%High oxidation states (low pH high anodic voltage, harsh oxidizing conditions) unexplored, need very specific structures with precise oxygen connectivity (aka high pressure \ce{SrIrO_3}) that can exist.
% Machine learning is the efficient way to explore this “exploring Antarctica for life” sparse space.
% __|

% __| %%%%%%%%%%%%%%%%%%%%%%%%%%%%%%%%%%%%%%%%%%%%%%%%%%%%%

% %%%%%%%%%%%%%%%%%%%%%%%%%%%%%%%%%%%%%%%%%%%%%%%%%%%%%%%%%
% ####################### Paragraph #######################
% %%%%%%%%%%%%%%%%%%%%%%%%%%%%%%%%%%%%%%%%%%%%%%%%%%%%%%%%%
% Paragraph outlining the structure of the paper
%
% %%%%%%%%%%%%%%%%%%%%%%%%%%%%%%%%%%%%%%%%%%%%%%%%%%%%%%%%%
% | - %%%%%%%%%%%%%%%%%%%%%%%%%%%%%%%%%%%%%%%%%%%%%%%%%%%%%
%
% \textbf{this needs a lead in sentence that transitions from the last paragraph}
This paper will be structured as follows,
first, we outline the generation of our candidate structures for \IrOtwo and \IrOthree and introduce the AL accelerated surrogate model.
%
Next, we demonstrate the application of our AL scheme to the \IrOtwo and \IrOthree prototype spaces and show the performance of our routine at finding the most stable polymorphs in the candidate sets.
%
Next, we analyze the crystallographic structural motifs of the DFT relaxed structures and discuss the trends between the most stable polymorphs.
%
Lastly, we construct a revised bulk Pourbaix diagram of the Ir-H$_2$O systems, and evaluate the OER performance for the most stable phases.
%
%highlighting the importance of the \IrOthree phases under OER.
% __| %%%%%%%%%%%%%%%%%%%%%%%%%%%%%%%%%%%%%%%%%%%%%%%%%%%%%
