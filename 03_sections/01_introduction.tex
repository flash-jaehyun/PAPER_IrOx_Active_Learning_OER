%%%%%%%%%%%%%%%%%%%%%%%%%%%%%%%%%%%%%%%%%%%%%%%%%%%%%%%%%%%%%%%%%%%%%%%%%%%%%%%
%% Introduction
%%
%% Key points:
%%   * Materials are composed of their "stoicheometry" and their "structure"
%%   * 3N dimensional Cartesian space of coordinates is prohibitelvey large
%%   * Our candidate space generation leverages symmetry to produce a set of
%%     structurally unique and well distributed (across the high dim. 3N space)
%%     polymorphs
%%   * Mention other methods for bulk crystal structure Discovery
%%     - simulated annealing
%%     - paper that Hammer put out (R58)
%%   * Potential energy surface (PES)
%%   * Global minimum (GM)
%%%%%%%%%%%%%%%%%%%%%%%%%%%%%%%%%%%%%%%%%%%%%%%%%%%%%%%%%%%%%%%%%%%%%%%%%%%%%%%


% ################################# Paragraph #################################
% %%%%%%%%%%%%%%%%%%%%%%%%%%%%%%%%%%%%%%%%%%%%%%%%%%%%%%%%%%%%%%%%%%%%%%%%%%%%%
% Setting up the general problem of crystal structure determination
% Very hard to exhaustively sample all structures,
% therefore it's hard to know if you are considering the global minimum structure
% All important properties of a material are governed by structure
% (assuming stoich is known)
% %%%%%%%%%%%%%%%%%%%%%%%%%%%%%%%%%%%%%%%%%%%%%%%%%%%%%%%%%%%%%%%%%%%%%%%%%%%%%
% | - Paragraph start
Predicting the thermodynamically favorable crystal structures for an arbitrary inorganic system remains a challenging problem in computational material science.
%
% Experimentally synthesized inorganic materials often conform to the global minimum energy structure (or similarly stable meta-stable phases).
%
When simulations are used to guide the search for new materials, the stable and meta-stable crystal structures must be known in order to predict the material properties.
% Structure vs composition enumeration (composition done to death)
% Give example of how over represented some motiffs are
% 188,000 systems with identical prototype ABO2 perovskite structure (~50 %)
Although there have been numerous examples in recent years of machine learning algorithms applied towards the prediction of formation energies of large ab-intio data sets, these data sets are relitively biased towards common structural motiffs.
%
For example, roughly half the entries (~200,000) in The Open Quantum Materials Database (OQMD) correspond to ternary-alloy combinations in the same close-packed cubic structure.
%have identical structures corresponding to ABO3 perovskites.
% Should we say that enumerating comp. is easy and struct. is hard?
As such, these efforts have been primarily concerned with the enumeration of composition and elemental identity and less so with the enumeration of structure.
%
The problem of thoroughly enumerating structural space is challenging for all but the most simple systems systems, because it involves sampling within the highly dimensional space of 3N cartesian coordinates, where N is the number of atoms in a given unit cell.
% COMBAK
This approach is only tractable for the most simple systems,
such as metallic crystals which tend to adopt highly symme

c close-packed configurations, but is less suited for anything more complex.
%
% I think less exposition about these structurall motiffs is better
This is particularly true for metal-oxides, a diverse and important class of materials,
which tend to organize themselves into well-defined local coordination environments such as octahedral and tetrahedral units (short-range order) which can arrange themselves in a large number of ways (long-range order).
%
% Traditional methods of crystal structure discovery, such as simulated annealing, operate within the high dimensional 3N cartesian space,
% making them scale poorly with system size.
%
Here we report on a crystal structure discovery algorithm that leverages machine learning surrogate models and an active learning framework to accelerate
the discovery of novel crystal structures in the Ir-O space.
%
The algorithm avoids operating in the highly dimensional 3N space by leveraging nature's propensity for symetry by preparing data sets with a large degree of structural diversity at fixed composition and stoicheometries.
%
% By considering only systems of unique stoicheomtry we leave the search space of 3N cartesions coordinates and enter a one dimensinal search space of structurally unique candidates.
%
% Within this search space, machine learning is used to accelarate and guide the search for promissing and stable structures by by-passing the need to perform expensive electronic structure calculations for the entire candidate space.

% | - __old__
% In order to navigate this vast space of materials, Machine Learning approaches has a huge potential to guide the search for structures of interest, and by-passing the need for expensive computations for all possible arrangements.

% Motivation for \ce{IrO_x}, low representation, longstanding controversy over oxidation states and topology, and demonstrates promise for OER and Li ion batteries.


% These basic building blocks can then be connected through 3 primary modes, vertex-sharing, edge-sharing, and face-sharing configurations (mid-range order) which can give rise to long-range order by forming porous or layered systems.
% Other classical crystal structure finding methods here
% The potential to form structures with different arrangements on multiple length-scales gives oxides a large degree of structural variety which it turn makes the space of possible polymorphs intractably large for traditional methods like simulated annealing, TEMP, or TEMP.
% __|

% __|


% ################################# Paragraph #################################
% %%%%%%%%%%%%%%%%%%%%%%%%%%%%%%%%%%%%%%%%%%%%%%%%%%%%%%%%%%%%%%%%%%%%%%%%%%%%%
% Talk about problem w.r.t. Iridium Oxides
% %%%%%%%%%%%%%%%%%%%%%%%%%%%%%%%%%%%%%%%%%%%%%%%%%%%%%%%%%%%%%%%%%%%%%%%%%%%%%
% | - Paragraph start
Herein, we focus on the chemical space of iridium oxide polymorphs,
which is an important class of materials with applications in electrochemistry. In particular, rutile-IrO2 (Ir in a [4+] oxidation state), the most stable form of iridium-oxide at standard conditions, is a well studied electrocatalyst for the oxygen evolution reaction (OER).
% (TODO cite science paper)
Previous studies on SrIrO3 electrocatayls for the OER demonstrated that Sr leaching left behind a highly oxidized Ir[6+] (IrO3) system that was predicted to be responsible for the observed OER activity.
%
For this reason we focused our study to search for stable polymorphs in the standard IrO2 stoicheometry as well as higher oxidation state corresponding to IrO3.
% TODO Reference the paper(s) from a-AlF3 systems
Additionally, IrO3 is a natural stoichiometry to target since it's the only one that can form octahedral units that are 100 percent corner sharing, and there is plenty of research on coordination crystals of the same stoicheometry. (TEMP reference).

% | - __old__
%Reported +6 oxidation state phases are achievable leading to high degree of structural variability, which is the highest for transition metals.
%High oxidation states (low pH high anodic voltage, harsh oxidizing conditions) unexplored, need very specific structures with precise oxygen connectivity (aka high pressure \ce{SrIrO_3}) that can exist.
% Machine learning is the efficient way to explore this “exploring Antarctica for life” sparse space.
% __|

% __|


% ################################# Paragraph #################################
% %%%%%%%%%%%%%%%%%%%%%%%%%%%%%%%%%%%%%%%%%%%%%%%%%%%%%%%%%%%%%%%%%%%%%%%%%%%%%
% Paragraph that starts to focus in on our actual algorithm
% Talk about the problem with optimizing in 3N cartesian space
% Best way to create a population of polymorphs is to consider symmetry
%   - nature loves symmetry
%   - Talk about other material databases
%   - They mostly explore composition space, despite this though, there does exist a lot of variety in structure
%       * QUESTION How much data do we have on the OQMD+MP structure uniqueness search
% %%%%%%%%%%%%%%%%%%%%%%%%%%%%%%%%%%%%%%%%%%%%%%%%%%%%%%%%%%%%%%%%%%%%%%%%%%%%%
% | - Paragraph start
% % 2  --> 3 key features (first is that we're not working in 3N space!!!)
% The two key features of our algorithm that make the exploration of an expansive space possible are its use of surrogate models and active learning framework.
% %
% Because DFT calculations are prohibitively computationally expensive to carry out for large data sets, herein we train a Gaussian Process (GP) machine learning model to serve as a surrogate model.
% %
% In general active learning applications, the requisite training data is not available.
% %
% Active learning frameworks are a means by which to generate the most valuable training data set, on the fly.
% __|





% | - __old__
% ################################# Paragraph #################################
% Crystallographic Discovery and Machine Learning.
% Current state of databases (OQMD, MP, CatHub, alfowlib).
% What parts of the database are missing (e.g. \ce{IrO_3}).
% Ankit - Condensed version of Ankit paper
% Prototyping databases to identify knowledge gaps.
% Deriving features from structures to describe heats of formation.
% What has been done (simple models) more recently using DFT+Machine Learning.


% ################################# Paragraph #################################
% Oxides for batteries/fuel cells, Iridium Oxide, OER, Lithiated \ce{IrO_3}.
% Highly oxidized phases of oxides for fuel cell and energy storage applications.
% __|
