% ################################# Paragraph #################################
Iterative Active Machine Learning and unique prototype identification to discover stable new materials and catalysts.
Motivation for \ce{IrO_x}, low representation, longstanding controversy over oxidation states and topology, and demonstrates promise for OER and Li ion batteries.

Reported +6 oxidation state phases are achievable leading to high degree of structural variability, which is the highest for transition metals.
High oxidation states (low pH high anodic voltage, harsh oxidizing conditions) unexplored, need very specific structures with precise oxygen connectivity (aka high pressure \ce{SrIrO_3}) that can exist.
Machine learning is the efficient way to explore this “exploring Antarctica for life” sparse space.
What we show here...


% ################################# Paragraph #################################
Crystallographic Discovery and Machine Learning.
Current state of databases (OQMD, MP, CatHub, alfowlib).
What parts of the database are missing (e.g. \ce{IrO_3}).
Ankit - Condensed version of Ankit paper
Prototyping databases to identify knowledge gaps.
% @Ankit
Deriving features from structures to describe heats of formation.
What has been done (simple models) more recently using DFT+Machine Learning.


% ################################# Paragraph #################################
Oxides for batteries/fuel cells, Iridium Oxide, OER, Lithiated \ce{IrO_3}.
Highly oxidized phases of oxides for fuel cell and energy storage applications.
