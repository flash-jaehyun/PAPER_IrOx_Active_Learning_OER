%%%%%%%%%%%%%%%%%%%%%%%%%%%%%%%%%%%%%%%%%%%%%%%%%%%%%%%%%%%%%%%%%%%%%%%%%%%%%%%
%% Introduction
%%
%%
%%%%%%%%%%%%%%%%%%%%%%%%%%%%%%%%%%%%%%%%%%%%%%%%%%%%%%%%%%%%%%%%%%%%%%%%%%%%%%%


% ################################# Paragraph #################################
% %%%%%%%%%%%%%%%%%%%%%%%%%%%%%%%%%%%%%%%%%%%%%%%%%%%%%%%%%%%%%%%%%%%%%%%%%%%%%
% Setting up the general problem of crystal structure determination
% Very hard to exhausitvely sample all structures,
% therefore it's hard to know if you are considering the global minimum strucutre
% All important properties of a material are governed by structure (assuming stoich is known)
% %%%%%%%%%%%%%%%%%%%%%%%%%%%%%%%%%%%%%%%%%%%%%%%%%%%%%%%%%%%%%%%%%%%%%%%%%%%%%
The general problem of solving for the most thermodynamically stable crystal structure for an arbitrary inorganic system remains a prohibitive challenge in material science.
Experimentally synthesized inorganic materials often conform to the globally minimum energy structure (or similarly stable meta-stable phases),
therefore it is desirable from a computational perspective to know the corresponding crystal structure to conduct simulations on.
This problem is theoretically challenging, as the number of possible crystal structures for a given stoichiometry is prohibitive.
Iterative Active Machine Learning and unique prototype identification to discover stable new materials and catalysts.
Motivation for \ce{IrO_x}, low representation, longstanding controversy over oxidation states and topology, and demonstrates promise for OER and Li ion batteries.

% ################################# Paragraph #################################
% %%%%%%%%%%%%%%%%%%%%%%%%%%%%%%%%%%%%%%%%%%%%%%%%%%%%%%%%%%%%%%%%%%%%%%%%%%%%%
% TEMP
% %%%%%%%%%%%%%%%%%%%%%%%%%%%%%%%%%%%%%%%%%%%%%%%%%%%%%%%%%%%%%%%%%%%%%%%%%%%%%
Reported +6 oxidation state phases are achievable leading to high degree of structural variability, which is the highest for transition metals.
High oxidation states (low pH high anodic voltage, harsh oxidizing conditions) unexplored, need very specific structures with precise oxygen connectivity (aka high pressure \ce{SrIrO_3}) that can exist.
Machine learning is the efficient way to explore this “exploring Antarctica for life” sparse space.
What we show here...

% ################################# Paragraph #################################
% %%%%%%%%%%%%%%%%%%%%%%%%%%%%%%%%%%%%%%%%%%%%%%%%%%%%%%%%%%%%%%%%%%%%%%%%%%%%%
% TEMP
% %%%%%%%%%%%%%%%%%%%%%%%%%%%%%%%%%%%%%%%%%%%%%%%%%%%%%%%%%%%%%%%%%%%%%%%%%%%%%
The two key features of our algorithm that make the exploration of an expansive space possible are its use of surrogate models and active learning framework.
Because DFT calculations are prohibitively computationally expensive to carry out for large data sets, herein we train a Gaussian Process (GP) machine learning model to serve as a surrogate model.
In general active learning applications, the requisite training data is not available.
Active learning frameworks are a means by which to generate the most valuable training data set, on the fly.


% ################################# Paragraph #################################
% Crystallographic Discovery and Machine Learning.
% Current state of databases (OQMD, MP, CatHub, alfowlib).
% What parts of the database are missing (e.g. \ce{IrO_3}).
% Ankit - Condensed version of Ankit paper
% Prototyping databases to identify knowledge gaps.
% Deriving features from structures to describe heats of formation.
% What has been done (simple models) more recently using DFT+Machine Learning.


% ################################# Paragraph #################################
% Oxides for batteries/fuel cells, Iridium Oxide, OER, Lithiated \ce{IrO_3}.
% Highly oxidized phases of oxides for fuel cell and energy storage applications.
