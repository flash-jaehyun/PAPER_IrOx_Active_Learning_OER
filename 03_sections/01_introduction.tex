%%%%%%%%%%%%%%%%%%%%%%%%%%%%%%%%%%%%%%%%%%%%%%%%%%%%%%%%%%%%%%%%%%%%%%%%%%%%%%%
% Introduction %%%%%%%%%%%%%%%%%%%%%%%%%%%%%%%%%%%%%%%%%%%%%%%%%%%%%%%%%%%%%%%%
% %%%%%%%%%%%%%%%%%%%%%%%%%%%%%%%%%%%%%%%%%%%%%%%%%%%%%%%%%%%%%%%%%%%%%%%%%%%%%
%
% Key points:
%   * Materials are composed of their "stoichiometry" and their "structure"
%   * 3N dimensional Cartesian space of coordinates is prohibitelvey large
%   * Our candidate space generation leverages symmetry to produce a set of
%     structurally unique and well distributed (across the high dim. 3N space)
%     polymorphs
%   * Mention other methods for bulk crystal structure Discovery
%     - simulated annealing
%     - paper that Hammer put out (R58)
%%%%%%%%%%%%%%%%%%%%%%%%%%%%%%%%%%%%%%%%%%%%%%%%%%%%%%%%%%%%%%%%%%%%%%%%%%%%%%%



% ################################# Paragraph #################################
% %%%%%%%%%%%%%%%%%%%%%%%%%%%%%%%%%%%%%%%%%%%%%%%%%%%%%%%%%%%%%%%%%%%%%%%%%%%%%
% * General intro into problem, give proper context
% * Make the point that structure has been all but ignored in recent years by the big-data/ML material science community (OQMD MP as examples)
%
% OUTLINE:
%  * Solving for crystal structures in a generally difficult problem
%  * It's an important problem to solve because all chemical/physical properties of a material are fundementally linked to the atomic structure.
%  * Although we've seen a lot of work in the past ~10 years from material science groups leveraging recent advances in big-data, ML, cheaper computational resources, high-throughput screening, materials databases, etc.
%    * Most of the focus has been on exploring compositionas and not structure
% %%%%%%%%%%%%%%%%%%%%%%%%%%%%%%%%%%%%%%%%%%%%%%%%%%%%%%%%%%%%%%%%%%%%%%%%%%%%%
% | - Paragraph start
% TODO this intro sentence will be a reference dump for literature in crystal structure prediction
Predicting the thermodynamically favorable crystal structures for an arbitrary inorganic system remains a challenging problem in computational material science.\cite{Woodley2008}
%
% Experimentally synthesized inorganic materials often conform to the global minimum energy structure (or similarly stable meta-stable phases).
%
% TODO Find reference to backup the importance of knowning the crystal structure to predict properties of real-world systems. I would consider this to be common knowledge but relevent article would be nice.
When simulations are used to guide the search for new materials, the stable and meta-stable crystal structures, \latin{i.e.} polymorphs,  above the convex hull of stability must be known in order to predict the material properties.
%
% Structure vs. composition enumeration (composition done to death)
% Give example of how over represented some motifs are
%
% TODO Cite multiple materials databases here
Although there have been numerous examples in recent years of machine learning algorithms applied towards the prediction of formation energies of large \latin{ab-initio} data sets,
these data sets are biased towards common structures and varying composition space.
%
For example, there are \mytilde\num{188000} entries in The Open Quantum Materials Database (OQMD) correspond to ternary-alloy combinations in the same close-packed cubic structure.\cite{Kirklin2015}
%
% Should we say that enumerating comp. is easy and struct. is hard?
As such, these efforts have been primarily concerned with the enumeration of composition (elemental identity and stoichiometries) and less so with the exploration of structural diversity.
%
% There are way more papers in the general area of GO than these
% I'm leaning away from framing as a "structural-diversity" problem with the databases and more so as a method of global optimization
% In the past, various groups have put forth methodologies to address the structural-diversity problem using computation for the MnO2~ and VOx~ polymorph spaces,
% but most of these methodologies are limited by the fact that they are operating within the highly intractable/dimensional PES.
%
% __|


% ################################# Paragraph #################################
% %%%%%%%%%%%%%%%%%%%%%%%%%%%%%%%%%%%%%%%%%%%%%%%%%%%%%%%%%%%%%%%%%%%%%%%%%%%%%
% * Introduction to global optimization and crystal structure predication community
% * Arguments and motivations for our unique approach
%
% OUTLINE:
%   * TEMP
% %%%%%%%%%%%%%%%%%%%%%%%%%%%%%%%%%%%%%%%%%%%%%%%%%%%%%%%%%%%%%%%%%%%%%%%%%%%%%
% | - Paragraph start
% TODO All global optimization papers go here
% TODO Add review paper for GO
Historically, the most common approach to crystal structure prediction relied on classical global optimization (GO) schemes,
in which the local/global minimum are found via numerical optimization routines that operate within the continuous potential energy surface (PES).%highly dimensional is also next sentence
%
% TODO Find solid example of shortcomings of classical GO schemes
These approaches, such as simulated annealing, are only tractable for the most simple systems,
such as metallic crystals which tend to adopt highly symmetric close-packed configurations, but is less suited for more complex materials.  % Example when it fails? RF Not off of the top of my head, it's a kind of claim that is generically true, but we should find something more concrete
%
This is because these methods are fundamentally limited by the curse of dimensionality associated with exploring the highly dimensional PES,
whose degrees of freedoms (and potential number of polymorphs) rises exponentially with system size.\cite{Stillinger1999}
%
The class of structurally diverse metal-oxides are a prime example of a chemical space for which structure prediction is non-trivial.
%
Metal-oxides are an important class of materials which tend to organize themselves into well-defined local coordination environments (octahedral, tetrahedral, etc.) which can assemble in a large variety of configurations with a long-range order.
%
% Emphasize the candidate space generation as a KEY feature of algorithm
To address the limitations of traditional global optimization methods,
here we report on a crystal structure discovery algorithm that leverages machine learning surrogate models and an active learning framework to accelerate the discovery of novel crystal structures at fixed composition.
%
% COMBAK Does the hansen2019atomistic paper really demonstrate structural optimizations? Double check
Active learning frameworks utilizing surrogate models have been demonstrated to successfully speed up materials discovery for alloy nanoparticles \cite{Jennings2019}, structural optimizations \cite{hansen2019atomistic}, and transition-state searches \cite{torres2019low}, as well as adaptive approaches for global optimization \cite{VanDenBossche2018}.
%
The algorithm avoids operating in the structural space by leveraging nature's propensity for symmetry by preparing data sets with a large degree of structural diversity at fixed composition.  % this sounds little hand waving | RF Maybe it needs more exposition, I'm convoluting 2 points here, that we don't do traditional GO opt. in cont. PES and that we use the intuition of symmetry to our advantage
%
As an alternative approach to GO, we first define the set of candidate crystal polymorphs, and secondly, search through the static list of candidates with a selection-type algorithm.
%
This approach relies on being able to prepare candidate structures that are likely to be physical and low in energy.
% TODO Find references for various ways of preparing candidate structures
To date, there have been various techniques to prepare candidate structures, including generating structures randomly, enumerating space groups, etc.
%
Empirically, we know that nature tends to favor symmetric structures, and thus herein, we use construct a dataset candidate structures that leverage this intuition.
%this looks  out of space. Consider moving up or eliminate. Also, this is more a discussion statememt.
% __|


% ################################# Paragraph #################################
% %%%%%%%%%%%%%%%%%%%%%%%%%%%%%%%%%%%%%%%%%%%%%%%%%%%%%%%%%%%%%%%%%%%%%%%%%%%%%
% Introduce specific system (Ir-oxides), relevance (OER), and what is to be
% gained by discovering new polymorphs
% %%%%%%%%%%%%%%%%%%%%%%%%%%%%%%%%%%%%%%%%%%%%%%%%%%%%%%%%%%%%%%%%%%%%%%%%%%%%%
% | - Paragraph start
Herein, we focus on the chemical space of iridium oxide polymorphs,
an important class of materials with applications in electrochemistry.
%
% See following link for list of important IrO2 OER papers:
% https://workflowy.com/#/e755141e2f66
In particular, \rIrOtwo (Ir[4+] oxidation state), is the most stable form  % best-known?
of iridium-oxide at standard conditions,
and is a well studied as one of the best electrocatalysts for the oxygen evolution reaction (OER).
\cite{Seitz2016,Lee2012a,McCrory2015,Trotochaud2012,Danilovic2014,Carmo2013,Miles1978,Beni1979}
%
Previous studies on SrIrO\textsubscript{3} electrocatalyst for the OER demonstrated that Sr leaching might leave behind a highly oxidized Ir (Ir[6+] for hypothetical \IrOthree) and it was argued as one possibly for observed high OER activity.\cite{Seitz2016}
%
% @Michal wants a Schlogl paper here, not sure which TODO
Other groups also observed such dissociation of \IrOx catalyst and subsequent formation of amorphous-like layer of unknown structure. \cite{Pearce2017}
%
Highly oxidized \IrOthree phases as also formed as the terminal structure of Li\textsubscript{x}IrO\textsubscript{3} anodes.\cite{Pearce2017}
%
% TODO Find reference for IrOxHy phases and their importance
For these reasons, we focused our study to search for stable polymorphs in the standard \IrOtwo stoicheometry and the more oxidized \IrOthree stoicheometry,
thereby neglecting the possibility of metal-hydroxide phases which have previously been shown to important.
%
Purely octahedral \IrOthree leads naturally to 100\% corner sharing octahedra,
where all terminal surface Ir-oxygens are potentially OER active sites.
%
% TODO Add references to octahedra literature
Furthermore, such pure corner sharing octahedral crystals are known from in other systems such fluorites and chlorites.


% | - __old__
%Reported +6 oxidation state phases are achievable leading to high degree of structural variability, which is the highest for transition metals.
%High oxidation states (low pH high anodic voltage, harsh oxidizing conditions) unexplored, need very specific structures with precise oxygen connectivity (aka high pressure \ce{SrIrO_3}) that can exist.
% Machine learning is the efficient way to explore this “exploring Antarctica for life” sparse space.
% __|

% __|


% ################################# Paragraph #################################
% %%%%%%%%%%%%%%%%%%%%%%%%%%%%%%%%%%%%%%%%%%%%%%%%%%%%%%%%%%%%%%%%%%%%%%%%%%%%%
% Paragraph outlining the structure of the paper
% %%%%%%%%%%%%%%%%%%%%%%%%%%%%%%%%%%%%%%%%%%%%%%%%%%%%%%%%%%%%%%%%%%%%%%%%%%%%%
% | - Paragraph start
%
In the first section, we define our prototype space and introduce the active-learning surrogate model.
%
Next, we highlight the application of AL to the \IrOtwo and \IrOthree prototype space.
%
Here we discuss the acceleration/performance and practical limitations of this approach as well as the nature of the most stable polymorphs.
%
Here, we also extract and analyze the rich structural information of our set.
%
In the section 3, we construct a revised bulk Pourbaix diagram of the Ir-H$_2$O system highlighting the importance of the \IrOthree phases under OER.
%
Finally, we construct thermodynamic OER volcano of most stable phases and discuss the trends in activities.
% __|







% | - __old__

% Very hard to exhaustively sample all structures, therefore it's hard to know if you are considering the global minimum structure


% | - __old__
% By considering only systems of unique stoicheometry we leave the search space of 3N Cartesian coordinates and enter a one dimensinal search space of structurally unique candidates.
%
% Within this search space, machine learning is used to accelerate and guide the search for promising and stable structures by by-passing the need to perform expensive electronic structure calculations for the entire candidate space.
% In order to navigate this vast space of materials, Machine Learning approaches has a huge potential to guide the search for structures of interest, and by-passing the need for expensive computations for all possible arrangements.
% Motivation for \ce{IrO_x}, low representation, longstanding controversy over oxidation states and topology, and demonstrates promise for OER and Li ion batteries.
% These basic building blocks can then be connected through 3 primary modes, vertex-sharing, edge-sharing, and face-sharing configurations (mid-range order) which can give rise to long-range order by forming porous or layered systems.
% Other classical crystal structure finding methods here
% The potential to form structures with different arrangements on multiple length-scales gives oxides a large degree of structural variety which it turn makes the space of possible polymorphs intractably large for traditional methods like simulated annealing, TEMP, or TEMP.
% __|


% ################################# Paragraph #################################
% %%%%%%%%%%%%%%%%%%%%%%%%%%%%%%%%%%%%%%%%%%%%%%%%%%%%%%%%%%%%%%%%%%%%%%%%%%%%%
% Paragraph that starts to focus in on our actual algorithm
% Talk about the problem with optimizing in 3N Cartesian space
% Best way to create a population of polymorphs is to consider symmetry
%   - nature loves symmetry
%   - Talk about other material databases
%   - They mostly explore composition space, despite this though, there does exist a lot of variety in structure
%       * QUESTION How much data do we have on the OQMD+MP structure uniqueness search
% %%%%%%%%%%%%%%%%%%%%%%%%%%%%%%%%%%%%%%%%%%%%%%%%%%%%%%%%%%%%%%%%%%%%%%%%%%%%%
% | - Paragraph start
% % 2  --> 3 key features (first is that we're not working in 3N space!!!)
% The two key features of our algorithm that make the exploration of an expansive space possible are its use of surrogate models and active learning framework.
% %
% Because DFT calculations are prohibitively computationally expensive to carry out for large data sets, herein we train a Gaussian Process (GP) machine learning model to serve as a surrogate model.
% %
% In general active learning applications, the requisite training data is not available.
% %
% __|


% ################################# Paragraph #################################
% Crystallographic Discovery and Machine Learning.
% Current state of databases (OQMD, MP, CatHub, alfowlib).
% What parts of the database are missing (e.g. \ce{IrO_3}).
% Ankit - Condensed version of Ankit paper
% Prototyping databases to identify knowledge gaps.
% Deriving features from structures to describe heats of formation.
% What has been done (simple models) more recently using DFT+Machine Learning.

% ################################# Paragraph #################################
% Oxides for batteries/fuel cells, Iridium Oxide, OER, Lithiated \ce{IrO_3}.
% Highly oxidized phases of oxides for fuel cell and energy storage applications.
% __|
