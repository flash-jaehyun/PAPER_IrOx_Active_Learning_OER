%%%%%%%%%%%%%%%%%%%%%%%%%%%%%%%%%%%%%%%%%%%%%%%%%%%%%%%%%%%%%%%%%%%%%%%%%%%%%%%
%% Introduction
%%
%% Key points:
%%   * Materials are composed of their "stoicheometry" and their "structure"
%%   * 3N dimensional Cartesian space of coordinates is prohibitelvey large
%%   * Our candidate space generation leverages symmetry to produce a set of
%%     structurally unique and well distributed (across the high dim. 3N space)
%%     polymorphs
%%   * Mention other methods for bulk crystal structure Discovery
%%     - simulated annealing
%%     - paper that Hammer put out (R58)
%%   * Potential energy surface (PES)
%%   * Global minimum (GM)
%%%%%%%%%%%%%%%%%%%%%%%%%%%%%%%%%%%%%%%%%%%%%%%%%%%%%%%%%%%%%%%%%%%%%%%%%%%%%%%


% ################################# Paragraph #################################
% %%%%%%%%%%%%%%%%%%%%%%%%%%%%%%%%%%%%%%%%%%%%%%%%%%%%%%%%%%%%%%%%%%%%%%%%%%%%%
% Setting up the general problem of crystal structure determination
% Very hard to exhaustively sample all structures,
% therefore it's hard to know if you are considering the global minimum structure
% All important properties of a material are governed by structure
% (assuming stoich is known)
% %%%%%%%%%%%%%%%%%%%%%%%%%%%%%%%%%%%%%%%%%%%%%%%%%%%%%%%%%%%%%%%%%%%%%%%%%%%%%
% | - Paragraph start
Predicting the thermodynamically favorable crystal structures for an arbitrary inorganic system remains a challenging problem in computational material science.\cite{}
%
% Experimentally synthesized inorganic materials often conform to the global minimum energy structure (or similarly stable meta-stable phases).
%
When simulations are used to guide the search for new materials, the stable and meta-stable crystal structures, i.e. polymorphs,  above the convex hull of stabilty  must be known in order to predict the material properties.\cite{}
% Structure vs composition enumeration (composition done to death)
% Give example of how over represented some motiffs are
% 188,000 systems with identical prototype ABO2 perovskite structure (~50 %)
Although there have been numerous examples in recent years of machine learning algorithms applied towards the prediction of formation energies of large ab-intio data sets, these data sets are biased towards common structures and varying composition space .\cite{}
%
For example, roughly half the entries (\~200,000) in The Open Quantum Materials Database (OQMD) correspond to ternary-alloy combinations in the same close-packed cubic structure.
%have identical structures corresponding to ABO3 perovskites.
% Should we say that enumerating comp. is easy and struct. is hard?
As such, these efforts have been primarily concerned with the enumeration of composition and elemental identity and less so with the enumeration of structure.
%
The problem of thoroughly enumerating structural space is challenging for all but the most simple systems systems, because it involves sampling within the highly dimensional space of $3N$-cartesian coordinates, where N is the number of atoms in a given unit cell.
% COMBAK
This approach is only tractable for the most simple systems,
such as metallic crystals which tend to adopt highly symmetric close-packed configurations, but is less suited for more complex materials such as metal oxides.
%
% I think less exposition about these structurall motiffs is better
This is particularly true for structurally diverse metal-oxides, a structurally diverse and important class of materials, which tend to organize themselves into well-defined local coordination environments such as octahedral and tetrahedral and other units (short-range order defining metal crystal field) arranged themselves in a large number of ways (long-range order, crystal symmetry group). Recently, several groups attempted to address the structural-diversity problem using computation for the MnO2~\cite{} and VOx~\cite{} polymorph space.

%
% Traditional methods of crystal structure discovery, such as simulated annealing, operate within the high dimensional 3N cartesian space,
% making them scale poorly with system size.
%
Here we report on a crystal structure discovery  algorithm that leverages machine learning surrogate models and an active learning framework to accelerate the discovery of novel crystal structures in the Ir-O space.
%
The algorithm avoids operating in the highly dimensional $3N$-space by leveraging nature's propensity for symmetry by preparing data sets with a large degree of structural diversity at fixed composition and stoichiometries.{\bf Are just trying to say that nature gives us sumetric strctures? }
%
% By considering only systems of unique stoicheomtry we leave the search space of 3N cartesions coordinates and enter a one dimensinal search space of structurally unique candidates.
%
% Within this search space, machine learning is used to accelarate and guide the search for promissing and stable structures by by-passing the need to perform expensive electronic structure calculations for the entire candidate space.

% | - __old__
% In order to navigate this vast space of materials, Machine Learning approaches has a huge potential to guide the search for structures of interest, and by-passing the need for expensive computations for all possible arrangements.

% Motivation for \ce{IrO_x}, low representation, longstanding controversy over oxidation states and topology, and demonstrates promise for OER and Li ion batteries.


% These basic building blocks can then be connected through 3 primary modes, vertex-sharing, edge-sharing, and face-sharing configurations (mid-range order) which can give rise to long-range order by forming porous or layered systems.
% Other classical crystal structure finding methods here
% The potential to form structures with different arrangements on multiple length-scales gives oxides a large degree of structural variety which it turn makes the space of possible polymorphs intractably large for traditional methods like simulated annealing, TEMP, or TEMP.
% __|

% __|


% ################################# Paragraph #################################
% %%%%%%%%%%%%%%%%%%%%%%%%%%%%%%%%%%%%%%%%%%%%%%%%%%%%%%%%%%%%%%%%%%%%%%%%%%%%%
% Talk about problem w.r.t. Iridium Oxides
% %%%%%%%%%%%%%%%%%%%%%%%%%%%%%%%%%%%%%%%%%%%%%%%%%%%%%%%%%%%%%%%%%%%%%%%%%%%%%
% | - Paragraph start
Herein, we focus on the chemical space of iridium oxide polymorphs,
which is an important class of materials with applications in electrochemistry. In particular, rutile-IrO$_2$ (Ir[4+] oxidation state), is the most stable form of iridium-oxide at standard conditions, and is a well studied electrocatalyst for the oxygen evolution reaction (OER).\cite{like 5 papers here pls}
Previous studies on SrIrO3 electrocatalyst for the OER demonstrated that Sr leaching might leave behind a highly oxidized Ir (Ir[6+] for hypotetical IrO$_3$) and it was argued as one possibly for observed high OER activity.\cite{Dickens} Other groups also observed such dissociation of IrOx catalyst and subsequent formation of amorphos-like layer of oknown strcuture\cite{Schlogl, beta-IrO3 OER paper}. Highly oxidized IrO$_3$ phases as also formed as the terminal struccture of LixIrO3 anodes.\cite{betaIrO3}

For above reasons, we focused our study to search for stable polymorphs in the standard IrO$_2$ stoichometry as well as higher oxidation state corresponding to IrO$_3$, but neglecting the possibility of mixed IrO$_2$(OH) phase.
Purely octaheadral IrO$_3$ leads naturally  to 100 percent corner sharing octahedra, where all terminal surface Ir-oxygens are potentially OER active sites. Furthermore, such pure corner sharing octaheadral crystals are known from in other sytems such flourites and chlorites. 

{\bf Do a short summary of the paper here mirroring abstract}. In the first section, we define our prototype space and introduce the active-learing surrogate model. Next, we highlight the application of AL to  IrO$_2$ and  IrO$_3$ prototype space. Here we discuss the acceleratiioon/performance and practical limitations of this approach as well as the nature of the most stable polymorphs. Here, we also extract and analyze the rich structural information of our set. In the section 3, we construct the new bulk Pourbaix diagram of the Ir-H$_2$O system highliting the imporance of the IrO$_3$ phases under OER. Finally, we construct termodynamic OER volcano of most stable phases and dissucss the trends in activities.         

% | - __old__
%these gray comments do not work form me. Just get rid of all this. 

%Reported +6 oxidation state phases are achievable leading to high degree of structural variability, which is the highest for transition metals.
%High oxidation states (low pH high anodic voltage, harsh oxidizing conditions) unexplored, need very specific structures with precise oxygen connectivity (aka high pressure \ce{SrIrO_3}) that can exist.
% Machine learning is the efficient way to explore this “exploring Antarctica for life” sparse space.
% __|

% __|


% ################################# Paragraph #################################
% %%%%%%%%%%%%%%%%%%%%%%%%%%%%%%%%%%%%%%%%%%%%%%%%%%%%%%%%%%%%%%%%%%%%%%%%%%%%%
% Paragraph that starts to focus in on our actual algorithm
% Talk about the problem with optimizing in 3N cartesian space
% Best way to create a population of polymorphs is to consider symmetry
%   - nature loves symmetry
%   - Talk about other material databases
%   - They mostly explore composition space, despite this though, there does exist a lot of variety in structure
%       * QUESTION How much data do we have on the OQMD+MP structure uniqueness search
% %%%%%%%%%%%%%%%%%%%%%%%%%%%%%%%%%%%%%%%%%%%%%%%%%%%%%%%%%%%%%%%%%%%%%%%%%%%%%
% | - Paragraph start
% % 2  --> 3 key features (first is that we're not working in 3N space!!!)
% The two key features of our algorithm that make the exploration of an expansive space possible are its use of surrogate models and active learning framework.
% %
% Because DFT calculations are prohibitively computationally expensive to carry out for large data sets, herein we train a Gaussian Process (GP) machine learning model to serve as a surrogate model.
% %
% In general active learning applications, the requisite training data is not available.
% %
% Active learning frameworks are a means by which to generate the most valuable training data set, on the fly.
% __|





% | - __old__
% ################################# Paragraph #################################
% Crystallographic Discovery and Machine Learning.
% Current state of databases (OQMD, MP, CatHub, alfowlib).
% What parts of the database are missing (e.g. \ce{IrO_3}).
% Ankit - Condensed version of Ankit paper
% Prototyping databases to identify knowledge gaps.
% Deriving features from structures to describe heats of formation.
% What has been done (simple models) more recently using DFT+Machine Learning.


% ################################# Paragraph #################################
% Oxides for batteries/fuel cells, Iridium Oxide, OER, Lithiated \ce{IrO_3}.
% Highly oxidized phases of oxides for fuel cell and energy storage applications.
% __|
