%%%%%%%%%%%%%%%%%%%%%%%%%%%%%%%%%%%%%%%%%%%%%%%%%%%%%%%%%%%%%%%%%%%%%%%%%%%%%%%
%% Supporting Information
%% TODO:
%%   - Gibbs corrections
%%   - OER mechanism
%%%%%%%%%%%%%%%%%%%%%%%%%%%%%%%%%%%%%%%%%%%%%%%%%%%%%%%%%%%%%%%%%%%%%%%%%%%%%%%


% | - Machine Learning Algorithm Methods
\subsection{Machine Learning Algorithm Methods}

Relevant details about the ML Gaussian process here % @Chris
% __|

% | - Electrochemical OER Computational Methods
\subsection{Electrochemical OER Computational Methods}

% | - Density Functional Theory Methods
\subsubsection{Density Functional Theory Methods}
% Get VASP #REF in the ex. word doc. that @Michal shared with me
All OER calculations were performed using density functional theory (DFT) implemented via the Vienna ab-initio simulation package (VASP) and utilizing the PBE exchange-correlation functional.
Dipole corrections were imposed on all non-symmetric slabs.
A 4x4x3 k-point mesh with gamma-point centered Monkshort-packing was used for all slabs.
The plane-wave energy cutoff was 500 eV.

% #COMBAK Figure out how much spacing was used for all slabs
% FIXME Change the A -> Angstrom symbol

% #COMBAK What kind of optimization routine was used? (Newtonian, BFGS?)
All slab calculations maintained a vacuum spacing of <15 A.
% #COMBAK change A -> Angstrom
All structures were relaxed utilizing a TEMP algorithm with a stop criteria being that all atoms satisfy a maximum force threshold of 0.02 eV/A.
% __|

% | - OER Thermodynamic Methodology
\subsubsection{OER Thermodynamic Methodology}
% __|

% | - Surface Energy Pourbaix Methodology
\subsubsection{Surface Energy Pourbaix Methodology}
Surface energy Pourbaix plots were constructed by calculating the surface energy of each slab by under standard conditions (V=0 and pH=0) and then utilizing the computational hydrogen electrode to compute the potential dependence of the surfaces.

Surface energy calculations were performed for various facets for slabs of increasing thickness.
The bulk energy was then extracted by fitting the total energy of the slabs against the number of layers as explained in REF2.
This was  done to avoid common issues of surface energy divergence associated with using a separate bulk energy calculation.

The sensitivity of a given slab to an applied bias is dependent on the composition of the surface,
in particular, the effect of coverage of electrolyte species which can deposit oxygen, hydrogen, and hydroxide species on the surface layers.
These additional O and H atoms are not referenced to the atoms in the slab, but are instead referenced to the computational hydrogen electrode and water-splitting reaction.
The equation for is as follows:
% TODO
% __|

% | - Bulk systems
% Formation energies of 4 polymorphs

% __|

% | - Table of OER energetics


% __|

Procedure:
- For the top/most stable bulk structures the following procedure was carried out

* Stable stoichiometric terminations were cut from the bulk Stable termination planes were guesstimated via intuition, and the x-ray diffraction pattern tool from Vesta

* Electrochemical surface coverage was elucidated via a surface Pourbaix analysis Need to know the coverage of surface under operating conditions (>1.23 V RHE)

* Thermodynamic/limiting potential analysis of the OER mechanistic pathway Volcano plot, limiting potentials, etc.

% __|
