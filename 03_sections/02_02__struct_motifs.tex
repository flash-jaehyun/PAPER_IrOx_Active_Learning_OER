% %%%%%%%%%%%%%%%%%%%%%%%%%%%%%%%%%%%%%%%%%%%%%%%%%%%%%%%%%
% | - Structural analysis of IrO2 and IrO3 oxides
% %%%%%%%%%%%%%%%%%%%%%%%%%%%%%%%%%%%%%%%%%%%%%%%%%%%%%%%%%
% TODO: Synthax for calling subplots in figure Figure 1.d or Figure 1.d)
% TODO: "we note that ..." is used a few times Kirsten
%
% Important Points:
%   * A good candidate set is a set of materials that have a large degree of 'structural diversity'
%
%   * We can make an analogy between the shape of these plots and typical van-der Waals curves
%     * I would argue that the tails are much less steep than a van-der Waals curve because in most cases the there is still a large degree of bonding
%       * Essentially these systems can create more and more 'porous' structures as the density is decreases, which allows us to get really large porous structures
%         * A good point can be made here that these types of systems could be useful for battery applications
%           * Pores are good in this context I think?
%
%
%   * The main coordination environments are 6 and 4-fold coordinated (Ir-O4/6 units)
%     * This is consistent with crystal field theory, or literature, etc.
%     * 6/4-coord accounts for TEMP percent of the IrO2/3 candidate space
%       * TODO Get these numbers
%     * There are in addition to 6/4-coord, a lot of other types of coordination environments
%       * A lot of these are simply weird, aphysical structures for sure
%         * Unassociated oxygens
%           * Singly-associated (part of N-hedra)
%           * Completely un-unassociated O atoms, just randomly placed in unit cell
%
%         * Especially at the extreme ends of the average coord metric
%       * But there are also a lot of legitimate coordination environments
%         * I've manually parsed the dataset and found
%
%   * There is a lot to say about the different kinds of 100% corner sharing octahedral IrO3 systems
%     * There are a lot of these systems that differ in subtle ways
%       * Basically octahedra can be "rotated" in different directions, making them distinct
%
%   * Interesting types of systems
%     * Layered systems (A lot of variety here)
%     * Cubic coordination environments (A few cool looking examples)
%     * New coordination environments
%       * TODO Remind myself again of these
%
%   * IrO2 has more 4-fold coordinated systems than IrO3
%     * Makes sense, the more oxygens you have the more oxygen-rich motifs are favored
%   * IrO2 has large "dip" in the EvsV "convex hull" while IrO3 has a much more shallow increase in energy as you move to the right from the most stable polymorph
%     * This is probably due to the fact that IrO3 can create more porous layered structures
%
% NOTES:
%   * Describe convex hull, classes of structures (\ce{$\alpha$-AlF3} like, rutile like, and layered, should be segregated in hull plot)
%   * Describe structures within each class, cite lit where appropriate
% __|
%%%%%%%%%%%%%%%%%%%%%%%%%%%%%%%%%%%%%%%%%%%%%%%%%%%%%%%%%%%



% %%%%%%%%%%%%%%%%%%%%%%%%%%%%%%%%%%%%%%%%%%%%%%%%%%%%%%%%%
% | - Intro to Plot, Metastability Defined
%
% __|
% %%%%%%%%%%%%%%%%%%%%%%%%%%%%%%%%%%%%%%%%%%%%%%%%%%%%%%%%%
% | - PARAGRAPH BODY
%
We next assess stability trends and structural variety of the DFT optimized structures, consisting of \num{384} and \num{191} unique \IrOtwo and \IrOthree polymorphs.
%
Figure~\ref{fig:E_vs_V}a,b shows the DFT computed \DHf for \IrOtwo and \IrOthree plotted against the inverse density,
a quantity that is sensitive to crystal porosity and connectivity.
%
To obtain a physically meaningful cutoff for \DHf, we computed the ``amorphous limit'' of Persson and coworkers for both \IrOtwo and \IrOthree,
which provides a stringent upper bound to polymorph synthesizability.~\cite{Aykol2018}
%
We found the amorphous limit for both \IrOtwo and \IrOthree to occur at a \DHf of \SI{-0.34}{\electronvolt}/atom and are displayed as horizontal lines in Figure~\ref{fig:E_vs_V}a,b.
%
There are \num{196} and \num{75} polymorphs for \IrOtwo and \IrOthree, respectively, that are within the amorphous synthesizability limit.
% __|
%%%%%%%%%%%%%%%%%%%%%%%%%%%%%%%%%%%%%%%%%%%%%%%%%%%%%%%%%%%


% =========================================================
% FIGURE ==================================================
% =========================================================
% | - Figure | Energy vs. Volume (motif distribution)
\begin{figure*}[!htb]
\centering
\makebox[\textwidth][c]{\includegraphics[width=\textwidth,height=\textheight,keepaspectratio]
% {02_figures/04_E_vs_V_coord/00_e_vs_v_coord_motiffs__v8__downsampled_1200x1200.pdf}}
{02_figures/04_E_vs_V_coord/00_e_vs_v_coord_motiffs__v9__downsampled_1200x1200__0600dpi.png}}
\caption{\label{fig:E_vs_V}
%
\DHf for the \num{384} \IrOtwo (a) and \num{191} \IrOthree (b) structurally unique DFT optimized structures in the candidate data set plotted against the volume per atom.
%
Insets in the low energy region for (a) and (b) are shown.
%
The color bar represents the average coordination number between Ir and O, with the most common, \num{6} (octahedra) and \num{4} (tetrahedral) coordinations, highlighted.
%
For \IrOtwo (c) and \IrOthree (d) we highlight the structures of select polymorphs.
%
The amorphous limits for \IrOtwo and \IrOthree (Figure \ref{table:amorph_limit}) defining a strict upper bound for synthesizability are displayed in (a) and (b) as horizontal lines.
}
\end{figure*}
% __| =====================================================
% =========================================================


% %%%%%%%%%%%%%%%%%%%%%%%%%%%%%%%%%%%%%%%%%%%%%%%%%%%%%%%%%
% | - Describe E vs V plot a bit
% TODO Say that the \IrOthree bulk phase corresponds to the o-covered regime
% There is a large variety in density on this plot
% IrO3 has a small dependance on the volume
% __|
% %%%%%%%%%%%%%%%%%%%%%%%%%%%%%%%%%%%%%%%%%%%%%%%%%%%%%%%%%
% | - PARAGRAPH BODY
%To increase the likelihood of discovering novel material types it is desirable that our candidate space is structurally diverse. Here we demonstrate that this is the case for our dataset, with
Computed materials span a large range of densities and coordination environments.
%
%Firstly there is a large variety in density for both compositions, such that both low volume (dense) and high volume (porous) structures are contained in the data set.
%
The lowest volume (highest density) structures correspond to an atomic packing factor of roughly \num{0.50}, which is where the most stable structures are found.
%
The highest volume (lowest density) systems sampled have atomic packing factors close to \num{0.15}.
%
However, for \IrOthree there is a comparatively weaker relationship between the energy and volume, such that even highly porous structures are within \SI{0.1}{\electronvolt}/atom of the most stable phase.
%
This is indicative of \IrOthree's high degree of polymorphism and ability to readily form layered and/or porous structures.
%
% COMBAK Explain this point better, this has to do with the oxidation state, coordination preservation rules that I've been playing around with
%
%This property is likely due to the fact that \IrOthree's oxidation state can more can form readily form layered or porous structures.
% __|
%%%%%%%%%%%%%%%%%%%%%%%%%%%%%%%%%%%%%%%%%%%%%%%%%%%%%%%%%%%


% %%%%%%%%%%%%%%%%%%%%%%%%%%%%%%%%%%%%%%%%%%%%%%%%%%%%%%%%%
% | - PARAGRAPH HEADER
% Explaining coordination motif distribution (octahedral, tetrahedral
% __|
% %%%%%%%%%%%%%%%%%%%%%%%%%%%%%%%%%%%%%%%%%%%%%%%%%%%%%%%%%
% | - PARAGRAPH BODY
%
Ir-O coordination environments were classified
(octahedral, square pyramidal, tetrahedral, cubic, etc.)
by using the chemEnv package, developed by Waroquiers et. al. \cite{Waroquiers2017} as implemented in the Pymatgen software \cite{Ong2013}.
%
Our dataset contains structures with coordination numbers ranging from 2 to 10,
with coordination numbers of six (octahedral, blue) and four (tetrahedral, red) being the most prevalent (see Figure~\ref{fig:E_vs_V}).
%
The vast majority of the most stable (within \SI{0.1}{\electronvolt}/atom) structures adopt an octahedral coordination environment,
a common coordination motif found to be favorable in many other transition metal oxides.\cite{Waroquiers2017}
%
% This has to be expanded upon or else maybe dropped, we can discuss
% I don't understand the end of this sentence. Give some information of how this applies to your structures.
% I've added a sentence at the end where we simply describe whether our most stable structures are corner vs. edge sharing
The arrangement of the octahedral units, which are connected through either corner- or edge-sharing octahedra,
can furthermore be used to classify the structures,
which typically have a combination of the two.
%
% Look into this deeper, there is probably a difference in fraction of mixed corner+edge between IrO2 and IrO3
Of the top ten \IrOtwo and \IrOthree structures, \num{9/10} of \IrOtwo and \num{5/10} of \IrOthree have a mixed corner- and edge-sharing octahedral packing.
%
This demonstrates that \IrOtwo prefers to form edge-sharing octahedra as a result of having to share more oxygens to maintain its stoichiometry.
%
\IrOthree has comparatively more oxygens per unit cell, and as such can adopt completely corner-sharing arrangements similar to cubic perovskite-type structures.
% __|
%%%%%%%%%%%%%%%%%%%%%%%%%%%%%%%%%%%%%%%%%%%%%%%%%%%%%%%%%%%


% %%%%%%%%%%%%%%%%%%%%%%%%%%%%%%%%%%%%%%%%%%%%%%%%%%%%%%%%%
% | - Talking about different structures
%
% __|
% %%%%%%%%%%%%%%%%%%%%%%%%%%%%%%%%%%%%%%%%%%%%%%%%%%%%%%%%%
% | - PARAGRAPH BODY
Figure~\ref{fig:E_vs_V}c,d shows a selection of metastable structures for \IrOtwo and \IrOthree, respectively.
%
For \IrOtwo, we reaffirm the rutile ground state.
%
% TODO Put in references for other pyrite formation enthalpies and confirm that these references cover experimental DHf values
Additionally, the experimentally synthesized high-pressure pyrite phase of \IrOtwo was found in our dataset and has a \DHf \mytilde\SI{0.1}{\electronvolt}/atom greater than rutile,
in agreement with theoretical and experimental calorimetric data.~\cite{bolzan1997structural, shirako2014synthesis}
%
% where only the rutile and pyrite phases are reported to have been synthesized \cite{bolzan1997structural, shirako2014synthesis}, the pyrite phase also predicted to be metastable in this work.
%
Several common \ABtwo crystal structures were found within the dataset,
including brookite-~\cite{brookite}, anatase-~\cite{anatase} and columbite-\IrOtwo phases.
%
For \IrOthree the eight most stable systems are reported in Figure~\ref{fig:iro3_al}, and labeled as (1)-(8) in Figure~\ref{fig:E_vs_V}b.
%
In addition to the most thermodynamically stable systems,
we have identified several interesting metastable structures,
including two dimensional (i), highly porous (iii) and one dimensional (v) polymorphs with varying degrees of porosity and connectivity,
which are important structural properties for applications as battery cathodes and ionic conductors.~\cite{Pearce2017,Pearce2019}
%
% \textbf{the labeling scheme jumping between two figures is a little confusing here, and it obfuscates the point I think you are trying to make in this last sentence}
% __|
%%%%%%%%%%%%%%%%%%%%%%%%%%%%%%%%%%%%%%%%%%%%%%%%%%%%%%%%%%%
