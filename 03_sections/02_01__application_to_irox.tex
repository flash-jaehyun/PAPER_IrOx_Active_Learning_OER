%%%%%%%%%%%%%%%%%%%%%%%%%%%%%%%%%%%%%%%%%%%%%%%%%%%%%%%%%%%
% | - Active Learning Machine Learning Results
% %%%%%%%%%%%%%%%%%%%%%%%%%%%%%%%%%%%%%%%%%%%%%%%%%%%%%%%%%
%
% Notes:
% Mention that we have ~15 metastable structures inbetween rutile and anatase IrO2
% __|
%%%%%%%%%%%%%%%%%%%%%%%%%%%%%%%%%%%%%%%%%%%%%%%%%%%%%%%%%%%



% %%%%%%%%%%%%%%%%%%%%%%%%%%%%%%%%%%%%%%%%%%%%%%%%%%%%%%%%%
% | - PARAGRAPH HEADER
% Intro/transition paragraph
% __|
% %%%%%%%%%%%%%%%%%%%%%%%%%%%%%%%%%%%%%%%%%%%%%%%%%%%%%%%%%
% | - PARAGRAPH BODY
%
The AL algorithm is now applied to the discovery of stable and unique polymorphs of \IrOtwo and \IrOthree.
%
Since the algorithm is applied to each stoichiometry individually,
we will only illustrate the results for \IrOthree,
a comparatively unstudied oxide system.
%
We will briefly touch on the results for \IrOtwo at the end and refer the reader to the Supporting Information for further details.
% __|
%%%%%%%%%%%%%%%%%%%%%%%%%%%%%%%%%%%%%%%%%%%%%%%%%%%%%%%%%%%


% %%%%%%%%%%%%%%%%%%%%%%%%%%%%%%%%%%%%%%%%%%%%%%%%%%%%%%%%%
% | - AL Results for IrO3
% * Introduce the convergence plots
% * The GP becomes more accurate as more DFT is acquired
% * The GP can start to recognize the low energy systems after minimal DFT
%
% We need to call them something else than "convergence plots" (bad name)
% __|
% %%%%%%%%%%%%%%%%%%%%%%%%%%%%%%%%%%%%%%%%%%%%%%%%%%%%%%%%%
% | - PARAGRAPH BODY
%
Figure~\ref{fig:iro3_al}a, shows a sequence of snapshots of the AL algorithm at different generations for \IrOthree.
%
% COMBAK Why are we sorting? Explain
Each subplot reports the predicted (hollow grey) and DFT-derived (solid red) formation enthalpies (\DHf) for each structure, sorted by stability.
%
As the algorithm acquires DFT data, the GP model's accuracy increases,
as evidenced by the decreasing uncertainties when comparing the initial and latter generations (\ref{fig:iro3_al}a.i-v).
%
At the top of each subplot of Figure~\ref{fig:iro3_al}a the identity of ten most stable polymorphs is tracked.
%
Initially, the ten most stable structures are randomly distributed across the entire candidate space due to the insufficiently trained GP model.
%
After only three generations (Figure~\ref{fig:iro3_al}a.ii) the GP model is sufficiently accurate to predict the ten most stable polymorphs as being low in energy.
%
By the fifth generation (\num{30} DFT relaxations) \num{4/10} of the most stable polymorphs have been acquired,
including the globally stable phase of \IrOthree.
%
% COMBAK Does this syntax work with num?
By the 14th generation of the AL
(Figure~\ref{fig:iro3_al}a.iv),
all but one of the top ten most stable structures have been acquired.
% __|
%%%%%%%%%%%%%%%%%%%%%%%%%%%%%%%%%%%%%%%%%%%%%%%%%%%%%%%%%%%


% %%%%%%%%%%%%%%%%%%%%%%%%%%%%%%%%%%%%%%%%%%%%%%%%%%%%%%%%%
% | - PARAGRAPH ABOUT STRUCTURES FOUND
% How to balance this with the next section that has more details on structural stuff?
% __|
% %%%%%%%%%%%%%%%%%%%%%%%%%%%%%%%%%%%%%%%%%%%%%%%%%%%%%%%%%
% | - PARAGRAPH BODY
%
% TEMP PARAGRAPH ON STRUCTURES FOUND
% __|
%%%%%%%%%%%%%%%%%%%%%%%%%%%%%%%%%%%%%%%%%%%%%%%%%%%%%%%%%%%


% %%%%%%%%%%%%%%%%%%%%%%%%%%%%%%%%%%%%%%%%%%%%%%%%%%%%%%%%%
% | - PARAGRAPH HEADER
% Performance of model
% __|
% %%%%%%%%%%%%%%%%%%%%%%%%%%%%%%%%%%%%%%%%%%%%%%%%%%%%%%%%%
% | - PARAGRAPH BODY
%
Figure~\ref{fig:iro3_al}c reports the number of the ten most stable systems acquired against the number of DFT calculations for the AL algorithm with the GP-UCB acquisition described previously and a random acquisition scheme to serve as a control.
%
The results of Figure~\ref{fig:iro3_al}c are averaged over 100 independent runs of AL algorithms with the 1 sigma standard deviation between these runs shown.
%
% COMBAK Pick better point to compare
Overall, the GP-UCB runs outperform the random acquisition runs, with only \num{50} DFT calculations on average needed to discover \num{7} of the \num{10} most stable systems.
%
This is compared to the \num{157} DFT calculations needed to discover \num{7/10} most stable polymorphs for the AL utilizing a random acquisition.
% __|
%%%%%%%%%%%%%%%%%%%%%%%%%%%%%%%%%%%%%%%%%%%%%%%%%%%%%%%%%%%


% =========================================================
% FIGURE ==================================================
% =========================================================
% | - Figure | IrO3 Convergence Plot
\begin{figure*}[!htb]
    \centering
    \makebox[\textwidth][c]{\includegraphics[width=\textwidth,height=\textheight,keepaspectratio]
        {02_figures/ml_convergence_plots/test_iro3_al.pdf}
    }
    \caption{\label{fig:iro3_al}
        % (a) %%%%%%%%%%%%%%%%%%%%%%%%%%%%%%%%%%%%%%%%%%%%%%%%%%%%%
        %
        (a) The state of the AL algorithm at five generations.
        %
        The enthalpy of formation per atom (\DHf) is plotted, ordered from most to least stable, against all \IrOthree candidates.
        %
        The number of DFT training points the GP model at each generation is displayed.
        %
        Hollow grey markers indicate a GP model predicted \DHf while red points indicate a DFT-computed quantity.
        %
        GP uncertainty estimates corresponding to one sigma are shown for all predictions.
        %
        At the top of subplots a.i through a.v, the x-axis positions of the ten most stable polymorphs are tracked at each generation by either red (acquired) or grey (not acquired) vertical lines.
        %
        Insets of the low energy region for each generation is displayed below each subplot.
        %
        The top ten most stable systems are colored and labeled to indicate their identity.
        % (b) %%%%%%%%%%%%%%%%%%%%%%%%%%%%%%%%%%%%%%%%%%%%%%%%%%%%%
        %
        (b) Crystal structures of the \num{8} most stable \IrOthree polymorphs.
        %
        The third most stable polymorphs is excluded since it is a combination of the first and second most stable systems.
        % (c) %%%%%%%%%%%%%%%%%%%%%%%%%%%%%%%%%%%%%%%%%%%%%%%%%%%%%
        %
        (c) The number of most stable \num{10} polymorphs of \IrOthree that are discovered as a function of the number of DFT bulk relaxations,
        averaged over \num{100} independent runs of the AL algorithm using the GP-UCB acquisition criteria (blue) and a random acquisition (grey).
        %
        The error bars indicate the standard deviation over \num{100} runs.
        %
        Red guide lines are displayed to show how many DFT calculations are needed to discover \num{7/10} of most stable polymorphs for the GP-UCB and random acquisition.
    }
\end{figure*}

% (b) %%%%%%%%%%%%%%%%%%%%%%%%%%%%%%%%%%%%%%%%%%%%%%%%%%%%%
% (b) Parity plot of the final ML models for \IrOtwo and \IrOthree predicting on either the pre-optimized (grey) or the post-optimized structures of \IrOtwo and \IrOthree.
% (c) %%%%%%%%%%%%%%%%%%%%%%%%%%%%%%%%%%%%%%%%%%%%%%%%%%%%%
% (c) Zoomed inset of the 6th generation of the AL loop.

% __| =====================================================
% =========================================================


% %%%%%%%%%%%%%%%%%%%%%%%%%%%%%%%%%%%%%%%%%%%%%%%%%%%%%%%%%
% | - PARAGRAPH HEADER
% Discussion on performance of the AL routine
% __|
% %%%%%%%%%%%%%%%%%%%%%%%%%%%%%%%%%%%%%%%%%%%%%%%%%%%%%%%%%
% | - PARAGRAPH BODY
We next evaluate the performance of the \IrOtwo and \IrOthree GP regression models trained on the full DFT dataset (corresponding to the final AL generation).
%
Figure~\ref{fig:parity} plots the GP model predicted \DHf against the DFT-computed values for two special cases.
%
Case 1) shows the predictions on the pre-optimized structural fingerprints, as is done in the regular operation of the algorithm when acquiring new structures and
case 2) shows the prediction of the same model onto the post-DFT optimized fingerprints (blue).
%
It is evident that the GP model is not accurate in quantitatively predicting the DFT formation energy of the candidate space using the pre-optimized fingerprints,
with a MAE of \mytilde\num{1.5} eV/atom.
%
The largest errors are highly skewed towards the unstable systems.
%
The same GP model does comparatively much better at predicting the formation energies of post-DFT optimized structures with an MAE of \mytilde\num{0.2} eV/atom,
roughly 100 meV/atom larger than model reported by Ward \latin{et al.}, which was trained on hundreds of thousands of DFT bulk calculations and reached an MAE of 0.8 eV/atom.
%
% __|
%%%%%%%%%%%%%%%%%%%%%%%%%%%%%%%%%%%%%%%%%%%%%%%%%%%%%%%%%%%


% %%%%%%%%%%%%%%%%%%%%%%%%%%%%%%%%%%%%%%%%%%%%%%%%%%%%%%%%%
% | - PARAGRAPH HEADER
% DISCUSS PERFORMANCE AND STRUCTURAL SHIFT
% __|
% %%%%%%%%%%%%%%%%%%%%%%%%%%%%%%%%%%%%%%%%%%%%%%%%%%%%%%%%%
% | - PARAGRAPH BODY
The fact that our models,
trained on orders of magnitude less data than those of Ward \latin{et al.},
achieve errors on par with them,
is due to the fact that we are training and predicting on a space whose properties are very narrowly constrained, while Ward \latin{et al.} is predicting on structures whose elements span the entire periodic table.
%
This drastic decrease in prediction error is not surprising since the post-DFT fingerprints directly corresponds to the target DFT energies.
%
Our comparison in Figure \ref{fig:parity} is to show that our model's inaccuracies are due to the large degree of structural drift that occurs during DFT relaxation,
the extent of which is not known \latin{a priori}.
%
Structures that are initialized in high energy configurations will therefore have high predicted \DHf,
and will then reconfigure into a more stable local configuration,
resulting in a large discrepancy between the pre-DFT predicted and final formation energies.
%
% COMBAK TODO \textbf{do determine the reason for this we did TEMP...}
Interestingly, despite the inaccuracy of predictions on unrelaxed structures,
the algorithm appears to perform well at discovering \num{7/10} of the most stable candidates after only \num{35} DFT calculations.
%
% COMBAK TODO \textbf{How do you know this?  This is in the section?  If it is you need to say it will be discussed in the next section here.}
The reason for this is that the pre-optimized structures that are similar enough to the most stable final equilibrium structures will not restructure considerably, meaning that their predicted formation energies will be close enough (and low enough) to be quickly picked up by the acquisition criteria.
% __|
%%%%%%%%%%%%%%%%%%%%%%%%%%%%%%%%%%%%%%%%%%%%%%%%%%%%%%%%%%%


% =========================================================
% FIGURE ==================================================
% | - Figure | IrO2/3 Parity Plot
\begin{figure*}[!htb]
\centering
\makebox[\textwidth][c]{\includegraphics[]
{02_figures/parity_plot.png}
}
\caption{\label{fig:parity}
%
Parity plot of the final ML models for \IrOtwo and \IrOthree predicting on either the pre-optimized (grey) or the post-optimized structures of \IrOtwo and \IrOthree.
}
\end{figure*}
% __|
% =========================================================
