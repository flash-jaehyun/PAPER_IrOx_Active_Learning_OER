%
Summary of OER results for the following four bulk structures of \IrOx: \rIrOtwo (green), \aIrOthree (purple), \rIrOthree (orange), and \bIrOthree (pink).
%
(a) Surface energy Pourbaix diagrams for each structure, with the surface energy of various facets and coverages shown as a function of applied potential (\VRHE).
%
% The coverage with bare surfaces (light lines), *OH covered (medium lines) and *O covered surfaces (thick lines) are shown.
%
The bulk Pourbaix diagram's bounds of stability at pH \num{0} are superimposed as horizontal bars at the bottom of each subplot.
%
The pseudo-stability regimes for the metastable \bIrOthree and \rIrOthree are indicated by dashed vertical lines.
%
(b) OER activity volcano for \IrOx systems considered utilizing the \DGOmOH{} descriptor.
%
The horizontal lines correspond to recent experimental OER limiting potentials for \rIrOtwo (110)~\cite{Kuo2017} and SrIrO\textsubscript{3}~\cite{Seitz2016},
at \SI[mode=text]{10}{\mA\per\cm\squared} (extrapolated values).
%
(c) Corresponding structural models for selected OER surfaces at one mono-layer O* coverage used for calculation of the overpotentials.
%
Color legend: oxygen (red), purple (iridium), coordination motif (white).
%
Computational cell is displayed by black lines.
%
All OER slab models and corresponding DFT energies are freely available under the ``FloresActive2020''~\cite{upload_CatHub} dataset at Catalysis-hub.org~\cite{Winther2019}.
