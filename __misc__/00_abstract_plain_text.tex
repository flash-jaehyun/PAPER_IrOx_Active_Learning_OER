The discovery of high-performing and stable materials for sustainable energy applications is a pressing goal in catalysis and materials science. Understanding the relationship between a material's structure and functionality is an important step in the process, such that viable polymorphs for a given chemical composition need to be identified. Machine-learning based surrogate models have the potential to accelerate the search for polymorphs that target specific applications. Herein, we report a readily generalizable active-learning (AL) accelerated algorithm for identification of electrochemically stable Iridium-oxide polymorphs of IrO2 and IrO3. The search is coupled to a subsequent analysis of the electrochemical stability of the discovered structures for the acidic oxygen evolution reaction (OER). The algorithm starts from all known AB2 and AB3 type structures (more than 38000) by identifying 956 unique materials structural prototypes. Next, we deploy the active-learning loop to  find 196 IrO2 polymorphs within the thermodynamic synthesizability limit and reaffirm the global stability of the rutile structure. We find 75 synthesizable IrO3 polymorphs and report a previously unknown FeF3-type structure as the most stable, termed as α-IrO3. To test model efficiency we compared to a random search via DFT. We find the AL algorithm more than doubles the rate of the discovery and acquires the most stable polymorphs in about 4.3 generations on average. Analysis of the structural properties of discovered polymorphs reveals that octahedral local coordination environments are preferred for nearly all low energy structures. Subsequent Pourbaix Ir-H2O analysis shows that α-IrO3 is the globally stable solid phase under acidic OER conditions and supersedes the stability of rutile IrO2. Calculation of theoretical OER surface activities reveal ideal weaker binding of the OER intermediates on α-IrO3 than on any other considered Iridium-oxide. The proposed AL algorithm  is easily generalizable to search for any binary metal-oxide structure with a predetermined stoichiometry.
