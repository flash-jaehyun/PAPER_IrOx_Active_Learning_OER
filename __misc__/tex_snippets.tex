% | - __misc__

  % | - Figures
  % \begin{figure}
  % \centering
  % \includegraphics[width=0.5\textwidth]{00_figures/oer_volcano.png}
  % \caption{\label{fig:benchmarking} TEMP}
  % \end{figure}
  % __|

  % | - Citing reference
  % \cite{
  %   Sham1983Density-FunctionalGap,
  %   Perdew1983PhysicalDiscontinuitiesb,
  %   Seidl1996GeneralizedProblem,
  %   Perdew2009DensityProblem
  %   }
  % __|

  % | - Citing figure in text
  % (Fig. \ref{fig:ochargeloc})
  % __|

  % | - Table
  % \begin{table}[h]
  % \centering
  % \caption{Free Energy Corrections for ORR Adsorbates on Cu/CTF Active Sites}
  % \label{dGcorrections}
  % \begin{tabular}{|c|c|c|}
  % \hline
  % \textbf{Adsorbate} &
  %     \textbf{
  %         \begin{tabular}[c]{@{}c@{}}Cu/CTF(6N)\\ dG - dE (eV)\end{tabular}
  %         } &
  %     \textbf{
  %         \begin{tabular}[c]{@{}c@{}}Cu/CTF(9N)\\ dG - dE (eV)\end{tabular}}
  %     \\ \hline
  % *OOH & 0.2 & 0.1 \\ \hline
  % *O & 0.0 & 0.0  \\ \hline
  % *OH & 0.2 & 0.2 \\ \hline
  % *$\rm{H_2O}$ & 0.6 & 0.5 \\ \hline
  %
  % \end{tabular}
  % \end{table}
  % __|

% __|


% | - __old__
% This is the place where we'll write about our results and discussions.
% We'll talk about the machine learning algorithm and the OER thermodynamics.


%
% \begin{equation}
%   O_2 +  {^*} + H^+ + e^- \rightleftharpoons {^*OOH}  \label{eqn:eqn1}
% \end{equation}
%
% This is a reference to \ref{eqn:eqn1}
%
% Things we need to to:
% \begin{itemize}
% \item Write the paper
% \item Finish the paper
% \item Finish writing the paper
% \item Submit paper
% \end{itemize}
% __|


% | - __old
% We have performed DFT+U calculations within the Vienna ab-initio simulation
% package1, 2 (VASP) using the projector augmented wave (PAW) potentials3.
% We adapt the PBE4 functional.
% together with the Hubbard-U method5 applied for the d-electrons of Ni
% (Ueff= 6.45 eV) and Fe (Ueff= 4.3 eV) atoms

% The bulk optimization calculations were performed on the 4x4x3 K-point mesh
% per 3x3x2 unit-cell of K8(Ni22Fe2)O48+ 16H2O and energy cutoff of 600 eV.
% For surfaces, we have employed 3x1 symmetric (100) slabs of 4 layers of M-O
% and 15 Å of vacuum.
% Here we used 4x4x1 k-point mesh and energy cutoff of 500 eV. All slabs were
% always fully relaxed below minimum threshold force of 0.02 eV/Å2.
% Finally, to obtain the theoretical overpotential for each surface, the Gibbs
% free energies of the OER intermediates within standard OER mechanism applied
% to many types of oxides   (*OH*, OH*O*, O*OOH*, OOH*O2(g)) by adding
% room temperature corrections to the potential energy, the zero point energy
% (ZPE) and the vibrational enthalpy and entropy contributions obtained by
% means of the harmonic approximation as published previously in Ref
% (cite 10.1021/jacs.7b02622).
% __|
